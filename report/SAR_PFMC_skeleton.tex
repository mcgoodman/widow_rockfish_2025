% Options for packages loaded elsewhere
\PassOptionsToPackage{unicode}{hyperref}
\PassOptionsToPackage{hyphens}{url}
\PassOptionsToPackage{dvipsnames,svgnames,x11names}{xcolor}
%
\documentclass[
]{scrartcl}

\usepackage{amsmath,amssymb}
\usepackage{iftex}
\ifPDFTeX
  \usepackage[T1]{fontenc}
  \usepackage[utf8]{inputenc}
  \usepackage{textcomp} % provide euro and other symbols
\else % if luatex or xetex
  \usepackage{unicode-math}
  \defaultfontfeatures{Scale=MatchLowercase}
  \defaultfontfeatures[\rmfamily]{Ligatures=TeX,Scale=1}
\fi
\usepackage{lmodern}
\ifPDFTeX\else  
    % xetex/luatex font selection
\fi
% Use upquote if available, for straight quotes in verbatim environments
\IfFileExists{upquote.sty}{\usepackage{upquote}}{}
\IfFileExists{microtype.sty}{% use microtype if available
  \usepackage[]{microtype}
  \UseMicrotypeSet[protrusion]{basicmath} % disable protrusion for tt fonts
}{}
\makeatletter
\@ifundefined{KOMAClassName}{% if non-KOMA class
  \IfFileExists{parskip.sty}{%
    \usepackage{parskip}
  }{% else
    \setlength{\parindent}{0pt}
    \setlength{\parskip}{6pt plus 2pt minus 1pt}}
}{% if KOMA class
  \KOMAoptions{parskip=half}}
\makeatother
\usepackage{xcolor}
\setlength{\emergencystretch}{3em} % prevent overfull lines
\setcounter{secnumdepth}{5}
% Make \paragraph and \subparagraph free-standing
\makeatletter
\ifx\paragraph\undefined\else
  \let\oldparagraph\paragraph
  \renewcommand{\paragraph}{
    \@ifstar
      \xxxParagraphStar
      \xxxParagraphNoStar
  }
  \newcommand{\xxxParagraphStar}[1]{\oldparagraph*{#1}\mbox{}}
  \newcommand{\xxxParagraphNoStar}[1]{\oldparagraph{#1}\mbox{}}
\fi
\ifx\subparagraph\undefined\else
  \let\oldsubparagraph\subparagraph
  \renewcommand{\subparagraph}{
    \@ifstar
      \xxxSubParagraphStar
      \xxxSubParagraphNoStar
  }
  \newcommand{\xxxSubParagraphStar}[1]{\oldsubparagraph*{#1}\mbox{}}
  \newcommand{\xxxSubParagraphNoStar}[1]{\oldsubparagraph{#1}\mbox{}}
\fi
\makeatother

\usepackage{color}
\usepackage{fancyvrb}
\newcommand{\VerbBar}{|}
\newcommand{\VERB}{\Verb[commandchars=\\\{\}]}
\DefineVerbatimEnvironment{Highlighting}{Verbatim}{commandchars=\\\{\}}
% Add ',fontsize=\small' for more characters per line
\usepackage{framed}
\definecolor{shadecolor}{RGB}{241,243,245}
\newenvironment{Shaded}{\begin{snugshade}}{\end{snugshade}}
\newcommand{\AlertTok}[1]{\textcolor[rgb]{0.68,0.00,0.00}{#1}}
\newcommand{\AnnotationTok}[1]{\textcolor[rgb]{0.37,0.37,0.37}{#1}}
\newcommand{\AttributeTok}[1]{\textcolor[rgb]{0.40,0.45,0.13}{#1}}
\newcommand{\BaseNTok}[1]{\textcolor[rgb]{0.68,0.00,0.00}{#1}}
\newcommand{\BuiltInTok}[1]{\textcolor[rgb]{0.00,0.23,0.31}{#1}}
\newcommand{\CharTok}[1]{\textcolor[rgb]{0.13,0.47,0.30}{#1}}
\newcommand{\CommentTok}[1]{\textcolor[rgb]{0.37,0.37,0.37}{#1}}
\newcommand{\CommentVarTok}[1]{\textcolor[rgb]{0.37,0.37,0.37}{\textit{#1}}}
\newcommand{\ConstantTok}[1]{\textcolor[rgb]{0.56,0.35,0.01}{#1}}
\newcommand{\ControlFlowTok}[1]{\textcolor[rgb]{0.00,0.23,0.31}{\textbf{#1}}}
\newcommand{\DataTypeTok}[1]{\textcolor[rgb]{0.68,0.00,0.00}{#1}}
\newcommand{\DecValTok}[1]{\textcolor[rgb]{0.68,0.00,0.00}{#1}}
\newcommand{\DocumentationTok}[1]{\textcolor[rgb]{0.37,0.37,0.37}{\textit{#1}}}
\newcommand{\ErrorTok}[1]{\textcolor[rgb]{0.68,0.00,0.00}{#1}}
\newcommand{\ExtensionTok}[1]{\textcolor[rgb]{0.00,0.23,0.31}{#1}}
\newcommand{\FloatTok}[1]{\textcolor[rgb]{0.68,0.00,0.00}{#1}}
\newcommand{\FunctionTok}[1]{\textcolor[rgb]{0.28,0.35,0.67}{#1}}
\newcommand{\ImportTok}[1]{\textcolor[rgb]{0.00,0.46,0.62}{#1}}
\newcommand{\InformationTok}[1]{\textcolor[rgb]{0.37,0.37,0.37}{#1}}
\newcommand{\KeywordTok}[1]{\textcolor[rgb]{0.00,0.23,0.31}{\textbf{#1}}}
\newcommand{\NormalTok}[1]{\textcolor[rgb]{0.00,0.23,0.31}{#1}}
\newcommand{\OperatorTok}[1]{\textcolor[rgb]{0.37,0.37,0.37}{#1}}
\newcommand{\OtherTok}[1]{\textcolor[rgb]{0.00,0.23,0.31}{#1}}
\newcommand{\PreprocessorTok}[1]{\textcolor[rgb]{0.68,0.00,0.00}{#1}}
\newcommand{\RegionMarkerTok}[1]{\textcolor[rgb]{0.00,0.23,0.31}{#1}}
\newcommand{\SpecialCharTok}[1]{\textcolor[rgb]{0.37,0.37,0.37}{#1}}
\newcommand{\SpecialStringTok}[1]{\textcolor[rgb]{0.13,0.47,0.30}{#1}}
\newcommand{\StringTok}[1]{\textcolor[rgb]{0.13,0.47,0.30}{#1}}
\newcommand{\VariableTok}[1]{\textcolor[rgb]{0.07,0.07,0.07}{#1}}
\newcommand{\VerbatimStringTok}[1]{\textcolor[rgb]{0.13,0.47,0.30}{#1}}
\newcommand{\WarningTok}[1]{\textcolor[rgb]{0.37,0.37,0.37}{\textit{#1}}}

\providecommand{\tightlist}{%
  \setlength{\itemsep}{0pt}\setlength{\parskip}{0pt}}\usepackage{longtable,booktabs,array}
\usepackage{calc} % for calculating minipage widths
% Correct order of tables after \paragraph or \subparagraph
\usepackage{etoolbox}
\makeatletter
\patchcmd\longtable{\par}{\if@noskipsec\mbox{}\fi\par}{}{}
\makeatother
% Allow footnotes in longtable head/foot
\IfFileExists{footnotehyper.sty}{\usepackage{footnotehyper}}{\usepackage{footnote}}
\makesavenoteenv{longtable}
\usepackage{graphicx}
\makeatletter
\def\maxwidth{\ifdim\Gin@nat@width>\linewidth\linewidth\else\Gin@nat@width\fi}
\def\maxheight{\ifdim\Gin@nat@height>\textheight\textheight\else\Gin@nat@height\fi}
\makeatother
% Scale images if necessary, so that they will not overflow the page
% margins by default, and it is still possible to overwrite the defaults
% using explicit options in \includegraphics[width, height, ...]{}
\setkeys{Gin}{width=\maxwidth,height=\maxheight,keepaspectratio}
% Set default figure placement to htbp
\makeatletter
\def\fps@figure{htbp}
\makeatother
% definitions for citeproc citations
\NewDocumentCommand\citeproctext{}{}
\NewDocumentCommand\citeproc{mm}{%
  \begingroup\def\citeproctext{#2}\cite{#1}\endgroup}
\makeatletter
 % allow citations to break across lines
 \let\@cite@ofmt\@firstofone
 % avoid brackets around text for \cite:
 \def\@biblabel#1{}
 \def\@cite#1#2{{#1\if@tempswa , #2\fi}}
\makeatother
\newlength{\cslhangindent}
\setlength{\cslhangindent}{1.5em}
\newlength{\csllabelwidth}
\setlength{\csllabelwidth}{3em}
\newenvironment{CSLReferences}[2] % #1 hanging-indent, #2 entry-spacing
 {\begin{list}{}{%
  \setlength{\itemindent}{0pt}
  \setlength{\leftmargin}{0pt}
  \setlength{\parsep}{0pt}
  % turn on hanging indent if param 1 is 1
  \ifodd #1
   \setlength{\leftmargin}{\cslhangindent}
   \setlength{\itemindent}{-1\cslhangindent}
  \fi
  % set entry spacing
  \setlength{\itemsep}{#2\baselineskip}}}
 {\end{list}}
\usepackage{calc}
\newcommand{\CSLBlock}[1]{\hfill\break\parbox[t]{\linewidth}{\strut\ignorespaces#1\strut}}
\newcommand{\CSLLeftMargin}[1]{\parbox[t]{\csllabelwidth}{\strut#1\strut}}
\newcommand{\CSLRightInline}[1]{\parbox[t]{\linewidth - \csllabelwidth}{\strut#1\strut}}
\newcommand{\CSLIndent}[1]{\hspace{\cslhangindent}#1}

\usepackage{hyphenat}
\usepackage{graphicx}
% and their extensions so you won't have to specify these with
 % every instance of \includegraphics
 \usepackage{pdfcomment}
\DeclareGraphicsExtensions{.pdf,.jpeg,.png}
\usepackage{wallpaper} % for the background image on title page
\usepackage{geometry}
% set font

% added by Ross
% % set font - - depends upon the driver
% \ifPDFTeX
%  %% only want this in body section headings and ToC, using \sf
%  \def\sfdefault{phv}% Helvetica instead of its clone Arial
%  \renewcommand{\sectfont}{\normalcolor
%   \def\bfdefault{bc}% bold condensed; i.e., narrow
%   \maybesffamily \bfseries }%% uses uhvb8ac
% % \def\sfdefault{lmss}% Latin Modern replaces Arial
% % \renewcommand{\sectfont}{\normalcolor
%  % \fontseries{sbc}\fontfamily{lmss}\selectfont }%% uses lmssdc10
% \else

\usepackage{fontspec}
\setsansfont[Ligatures=TeX]{Arial Narrow}

% added by Ross
%\fi
%\usepackage[scaled=0.9]{helvet}% needed later to replace Arial Narrow
\usepackage[headsepline=0.005pt:,footsepline=0.005pt:,plainfootsepline,automark]{scrlayer-scrpage}
\clearpairofpagestyles
\ohead[]{\headmark} \cofoot[\pagemark]{\pagemark}
\ModifyLayer[addvoffset=-.6ex]{scrheadings.foot.above.line}
\ModifyLayer[addvoffset=-.6ex]{plain.scrheadings.foot.above.line}
\setkomafont{pageheadfoot}{\small}
% Landscape tables and figures
\usepackage{pdflscape}
\newcommand{\blandscape}{\begin{landscape}}
\newcommand{\elandscape}{\end{landscape}}

% Acronyms
\usepackage[acronym]{glossaries}
\glsdisablehyper
\loadglsentries{sa4ss_glossaries.tex}
\usepackage{booktabs}
\usepackage{longtable}
\usepackage{array}
\usepackage{multirow}
\usepackage{wrapfig}
\usepackage{float}
\usepackage{colortbl}
\usepackage{pdflscape}
\usepackage{tabu}
\usepackage{threeparttable}
\usepackage{threeparttablex}
\usepackage[normalem]{ulem}
\usepackage{makecell}
\usepackage{xcolor}
\usepackage{fontspec}
\usepackage{multicol}
\usepackage{hhline}
\newlength\Oldarrayrulewidth
\newlength\Oldtabcolsep
\usepackage{hyperref}
\lohead{SPECIES assessment 2025}
\makeatletter
\@ifpackageloaded{caption}{}{\usepackage{caption}}
\AtBeginDocument{%
\ifdefined\contentsname
  \renewcommand*\contentsname{Table of contents}
\else
  \newcommand\contentsname{Table of contents}
\fi
\ifdefined\listfigurename
  \renewcommand*\listfigurename{List of Figures}
\else
  \newcommand\listfigurename{List of Figures}
\fi
\ifdefined\listtablename
  \renewcommand*\listtablename{List of Tables}
\else
  \newcommand\listtablename{List of Tables}
\fi
\ifdefined\figurename
  \renewcommand*\figurename{Figure}
\else
  \newcommand\figurename{Figure}
\fi
\ifdefined\tablename
  \renewcommand*\tablename{Table}
\else
  \newcommand\tablename{Table}
\fi
}
\@ifpackageloaded{float}{}{\usepackage{float}}
\floatstyle{ruled}
\@ifundefined{c@chapter}{\newfloat{codelisting}{h}{lop}}{\newfloat{codelisting}{h}{lop}[chapter]}
\floatname{codelisting}{Listing}
\newcommand*\listoflistings{\listof{codelisting}{List of Listings}}
\makeatother
\makeatletter
\makeatother
\makeatletter
\@ifpackageloaded{caption}{}{\usepackage{caption}}
\@ifpackageloaded{subcaption}{}{\usepackage{subcaption}}
\makeatother

\ifLuaTeX
\usepackage[bidi=basic]{babel}
\else
\usepackage[bidi=default]{babel}
\fi
\babelprovide[main,import]{english}
% get rid of language-specific shorthands (see #6817):
\let\LanguageShortHands\languageshorthands
\def\languageshorthands#1{}
\ifLuaTeX
  \usepackage{selnolig}  % disable illegal ligatures
\fi
\usepackage{bookmark}

\IfFileExists{xurl.sty}{\usepackage{xurl}}{} % add URL line breaks if available
\urlstyle{same} % disable monospaced font for URLs
\hypersetup{
  pdftitle={Status of SPECIES off the U.S. West Coast in 2025},
  pdfauthor={Michael Kinneen},
  pdflang={en},
  colorlinks=true,
  linkcolor={blue},
  filecolor={Maroon},
  citecolor={Blue},
  urlcolor={Blue},
  pdfcreator={LaTeX via pandoc}}


\title{Status of SPECIES off the U.S. West Coast in 2025}
\author{Michael Kinneen}
\date{2025-05-11}

\begin{document}
  \begin{titlepage}
  % This is a combination of Pandoc templating and LaTeX
  % Pandoc templating https://pandoc.org/MANUAL.html#templates
  % See the README for help

  \newgeometry{top=2in,bottom=1in,right=1in,left=1in}
  \begin{minipage}[b][\textheight][s]{\textwidth}
  % Ross would've subbed lines 6, 8 with these lines:
  %\newgeometry{top=2in,bottom=1in,right=1in,left=1in}%
  %\noindent  %\tracingall
  %\begin{minipage}[b][\textheight][s]{.975\textwidth}%% RRM: avoid Overfull box


  \raggedright

  % \includegraphics[width=2cm]{NOAA_Transparent_Logo.png}

  % background image


  % Title and subtitle
  {\huge\bfseries\nohyphens{Status of SPECIES off the U.S. West Coast in
  2025}}\\[1\baselineskip]
  % Ross would change the end of the above line to the following because \par must come before the group closes and line-depth reverts.
  % }\par}%\\[1\baselineskip]



  \vspace{1\baselineskip}
  % Ross would change this to 2\baselineskip

  %%%%%% Cover image

  \vspace{1\baselineskip}

  % Authors
  % This hairy bit of code is just to get "and" between the last 2
  % authors. See below if you don't need that
  %
  {\large{Michael Kinneen}}%
  %
  {\textsuperscript{1}}%
  %


  % This is how to do it if you don't need the "and"

  %%%%%% Affiliations
  \vspace{2\baselineskip}

  \hangindent=1em
  \hangafter=1
  % Ross would change the above line to:
  % \hangafter=1\relax
  %
  {1}.~{NOAA Fisheries Northwest Fisheries Science Center}%
  %
  %
  % Ross recommends putting address on one line
  , %
  {2725 Montlake Boulevard East}%
  %


  %%%%%% Correspondence
  \vspace{1\baselineskip}


  %use \vfill instead to get the space to fill flexibly
  %\vspace{0.25\textheight} % Whitespace between the title block and the publisher

  \vfill


  % Whitespace between the title block and the tagline
  \vspace{1\baselineskip}

  %%%%%% Tagline at bottom
  % Ross says the tagline below could also be centered
  \includegraphics[alt={},width=2cm]{support_files/us_doc_logo.png}\newline % empty curly brackets without alt text is suitable for this logo because it's purely decorative/an "artifact"
  U.S. Department of Commerce\newline
  National Oceanic and Atmospheric Administration\newline
  National Marine Fisheries Service\newline
  Northwest Fisheries Science Center\newline

  \end{minipage}
  \restoregeometry
  \end{titlepage}

\renewcommand*\contentsname{Table of contents}
{
\hypersetup{linkcolor=}
\setcounter{tocdepth}{3}
\tableofcontents
}

\newpage{}

Please cite this publication as:

Michael Kinneen, Maurice Goodman, . (2025) Status of Widow Rockfish off
the U.S. West Coast in 2025. Pacific Fishery Management Council.
{[}XX{]} p.

\newpage{}

\pagenumbering{roman}
\setcounter{page}{1}

\renewcommand{\thetable}{\roman{table}}
\renewcommand{\thefigure}{\roman{figure}}

\section*{Disclaimer}\label{disclaimer}
\addcontentsline{toc}{section}{Disclaimer}

These materials do not constitute a formal publication and are for
information only. They are in a pre-review, pre-decisional state and
should not be formally cited or reproduced. They are to be considered
provisional and do not represent any determination or policy of NOAA or
the Department of Commerce.

\newpage{}

\section*{Executive Summary}\label{executive-summary}
\addcontentsline{toc}{section}{Executive Summary}

Checking to see if this works \gls{s-tri}

\subsection*{Stock}\label{stock}
\addcontentsline{toc}{subsection}{Stock}

\subsection*{Catches}\label{catches}
\addcontentsline{toc}{subsection}{Catches}

\subsection*{Data and Assessment}\label{data-and-assessment}
\addcontentsline{toc}{subsection}{Data and Assessment}

\subsection*{Stock biomass and
dynamics}\label{stock-biomass-and-dynamics}
\addcontentsline{toc}{subsection}{Stock biomass and dynamics}

\subsection*{Recruitment}\label{recruitment}
\addcontentsline{toc}{subsection}{Recruitment}

\subsection*{Exploitation status}\label{exploitation-status}
\addcontentsline{toc}{subsection}{Exploitation status}

\subsection*{Ecosystem considerations}\label{ecosystem-considerations}
\addcontentsline{toc}{subsection}{Ecosystem considerations}

\subsection*{Reference points}\label{reference-points}
\addcontentsline{toc}{subsection}{Reference points}

\subsection*{Management performance}\label{management-performance}
\addcontentsline{toc}{subsection}{Management performance}

\subsection*{Harvest projections}\label{harvest-projections}
\addcontentsline{toc}{subsection}{Harvest projections}

\subsection*{Decision table}\label{decision-table}
\addcontentsline{toc}{subsection}{Decision table}

\subsection*{Scientific uncertainty}\label{scientific-uncertainty}
\addcontentsline{toc}{subsection}{Scientific uncertainty}

\subsection*{Research and data needs}\label{research-and-data-needs}
\addcontentsline{toc}{subsection}{Research and data needs}

\subsection*{Rebuilding projections}\label{rebuilding-projections}
\addcontentsline{toc}{subsection}{Rebuilding projections}

\newpage{}

\setlength{\parskip}{5mm plus1mm minus1mm}
\pagenumbering{arabic}
\setcounter{page}{1}
\setcounter{section}{0}
\renewcommand{\thefigure}{\arabic{figure}}
\renewcommand{\thetable}{\arabic{table}}
\setcounter{table}{0}
\setcounter{figure}{0}

\section{Introduction}\label{introduction}

\emph{Sebastes entomelas} (Widow Rockfish) is named after its
black-lined gut cavity (\emph{ento} meaning within and \emph{melas}
meaning black). It has been referred to as buda, beccafico (Italian
bird), and viuva (widow) prior to the 1930s. More recently, the Widow
Rockfish is also called brownie, belinda bass, brown bomber, and soft
brown.

This is an assessment of Widow Rockfish that inhabit the waters off
California, Oregon, and Washington from the U.S.-Canadian border in the
north to the U.S.-Mexico border in the south, and does not include Puget
Sound waters (Figure 1). This assessment represents a thorough
reconsideration of the data, data preparation, and model structure for
assessing Widow Rockfish, including reinvestigations of recent and
historical catches (including discards), length and age data, and fleet
structure.

\subsection{Distribution and Stock
Structure}\label{distribution-and-stock-structure}

Widow Rockfish inhabit water depths of 25--370 m from northern Baja
California, Mexico to Southeastern Alaska, and are most abundant from
British Columbia to Northern California. Although catches north of the
U.S.-Canada border or south of the U.S.-Mexico border were not included
in this assessment, it is possible that these populations contribute to
the biomass of Widow Rockfish off of the U.S. West Coast through adult
migration and/or larval dispersion.

There is little evidence of genetically separate stocks along the U.S.
coast and past assessments have used a single area, coastwide model with
multiple fisheries (He et al. 2011). In 2011, a two-area assessment
model was brought forward for review, and was found to be similar to a
coastwide model (He et al. 2011). There is some evidence of biological
differences between areas. For example, Widow Rockfish collected off
California tend to mature at a smaller length than Widow Rockfish
collected off of Oregon (\textbf{barss\_maturity\_1987?}). This may be
due to environmental or anthropogenic effects rather than genetic
differences. The connectivity of Widow Rockfish populations throughout
its range is unknown and it was decided to continue with a single area
model for this assessment instead of potentially lose prediction power
by splitting the data into two separate areas.

\subsection{Life History and Ecosystem
Interactions}\label{life-history-and-ecosystem-interactions}

Widow Rockfish are atypical for West Coast rockfish species because they
form dense midwater aggregations at night, which were largely undetected
until the late 1970s. They are typically found over high relief strata
and near cobblestone. The diet of Widow Rockfish is dominated by species
that comprise the deep scattering layers, including salps, myctophids,
\emph{Sergestes similis} (a caridean shrimp), and euphausiids (Adams
1987).

Widow Rockfish are ovoviviparous with gestation lasting from 1 to 3
months. Parturition occurs earlier in southern latitudes (December-March
off California) than in northern latitudes (April in British Columbia)
and occur once a year (\textbf{barss\_maturity\_1987?}). Estimates of
fecundity of Widow Rockfish range from 95,375 oocytes at 33 cm to
1,113,000 oocytes at 52 cm (\textbf{Boehlert\_fecundity\_1982?}).

There is little information regarding the movement of Widow Rockfish.
Past assessments have assumed a two-area model because of differences in
growth and maturity (see (He et al. 2011)). However, using recent
observations from the NWFSC shelf/slope survey to follow two separate
cohorts through time and space suggests that Widow Rockfish may recruit
in the south and disperse northward as they age (Figure 2). Spatial
recruitment and movement patterns of Widow Rockfish are uncertain and
much more investigation and sampling is needed to fully understand them.

\subsection{Fishery description}\label{fishery-description}

Widow Rockfish were lightly exploited by bottom trawl and hook-and-line
gears prior to the 1980s. After many attempts to start trawl fisheries
off the west coast of the United States in the late 1800s, the
availability of otter trawl nets and the diesel engine in the mid-1920s
helped trawl fisheries expand (Douglas 1998). The trawl fisheries really
became established during World War II when demand increased for shark
livers and bottomfish. A mink food fishery also developed during World
War II (Jones and Harry Jr 1960). Foreign fleets began fishing for
rockfish in the mid-1960s until the EEZ was implemented in 1977 (Rogers
2003). Longline catches of Widow Rockfish are present from the turn of
the century and continue in recent years, mainly from fisheries
targeting sablefish and halibut.

In the late 1960s and early 1970s, it is reported that foreign fishing
vessels caught large numbers of Widow Rockfish (Rogers 2003). In the
late 1970s a domestic midwater trawl fishery began developing off of
Oregon when it was realized that Widow Rockfish form dense aggregations
at night (\textbf{Gunderson\_great\_1984?}). The fishery expanded very
quickly, with landings from trawl, net, and hook-and-line gears
increasing more than 20 times by the early 1980s (Table 1). As early as
1982, trip limits were imposed to keep catches below recommended annual
levels (Table 3). Trip limits became more restrictive over the years
until Widow Rockfish was declared overfished in 2001. In 2002, harvest
guidelines were greatly reduced and over the last decade have been
small, although increasing since 2004 (Table 4).

Historical discarding practices are not well known, but it is believed
that little discarding occurred prior to management restrictions. With
the introduction of trip limits, limited data from the mid-1980s show
occasional very high discard rates of Widow Rockfish from tows that
occurred near the end of a trip.

More detailed information of the fisheries in each state is given in
Section 2.2.1 where the reconstructed landings are discussed.

\subsection{Management History and
Performance}\label{management-history-and-performance}

Widow Rockfish has been a small large component of groundfish fisheries
since the late 1970s. The landings of Widow Rockfish have been
historically governed by harvest guidelines and trip limits, while
recently management is imposed with total catch harvest limits in the
form of overfishing limits (OFLs), acceptable biological catches (ABCs),
and annual catch limits (ACLs). A trawl rationalization program,
consisting of an individual fishing quota (IFQ) or catch shares system
was implemented in 2011 for the limited entry trawl fleet targeting
non-whiting groundfish, including Widow Rockfish, and the trawl fleet
targeting and delivering whiting to shore-based processors. The limited
entry at-sea trawl sectors (motherships and catch-processors) that
target whiting and process at sea are managed in a system of harvest
cooperatives.

Limits on Widow Rockfish were first established in 1982 (Table 3). These
were implemented as trip limits and cumulative landing limits that were
first imposed by trip, then week, then every 2 weeks, month, 2 months,
and eventually into periods. In many years, the trip limits on Widow
Rockfish were significantly reduced at the end of the year to avoid
exceeding the harvest recommendations. Some important years were 1985
when trip limits were reduced to 30,000 pounds once per week or 60,000
pounds once every 2 weeks, 1990 when trip limits were reduced to 15,000
or 25,000 pounds every one or two weeks, respectively, 1998 when a
25,000 pound cumulative limit per two-month period was implemented, and
2011 when catch shares was implemented.

A sorting requirement was implemented for Widow Rockfish in the early
1980s with California beginning in 1982, Oregon in 1984, and Washington
in 1988. Some important events that could affect fishery selectivity are
the gear restrictions implemented in 2000, implementation of Rockfish
Conservation Areas (RCA's) in 2002, seasonal changes to the RCA's in
2007, and the beginning of catch shares in 2011.

Table 4 shows that recent landings have been below recommended catch
levels. Landings are a considerable amount below the ACL, and it is
unlikely that total mortality has exceeded the ACL in the last 10 years.

\subsection{Fisheries off Canada and
Alaska}\label{fisheries-off-canada-and-alaska}

Widow Rockfish are distributed throughout Canada and Southeast Alaska
and are commonly caught in trawl and hook-and-line fisheries. However,
the landings from the fisheries in these areas are estimated to harvest
Widow Rockfish at much smaller rate than has been observed off
California, Oregon, and Washington mostly due to lower abundance of
Widow Rockfish, but also partly due to precautionary behavior of
Canadian managers after the large catches followed by management
restrictions and concerns of the U.S. fishery in the early 1980s.

Alaska formed the ``Other Rockfish'' complex in 2012 from the
combination of Other Slope Rockfish and the Widow and Yellowtail
Rockfishes from the Pelagic Shelf Rockfish category. This new complex
includes 18 species and Widow Rockfish are a small proportion of the
catch (less than 5\%). Total biomass estimates are provided by the Gulf
of Alaska (GOA) triennial/biennial trawl survey. ABC's and OFL's were
set for the Other Rockfish Complex and component species in 2013 with a
recommended OFL in 2014 of 5,347 mt for the complex. Widow Rockfish
comprise a small part of this complex in Alaska.

The fishery for Widow Rockfish in British Columbia, Canada started in
1986 although some very small landings occurred in the mid-1970s.
Landings peaked at about 4,500 mt in 1990 and were around 2,000 mt
throughout the 1990s {[}dfo\_widow\_1999{]}. Most landings occurred in a
midwater trawl fishery, but there have also been reports of ``nuisance
catches in the salmon troll fishery''. An assessment of Widow Rockfish
in Canada was completed in 1998 (\textbf{stanley\_shelf\_1999?}) as part
of a shelf rockfish complex. Additional research has since been done on
the estimation of biomass of particular aggregations of Widow Rockfish
(Stanley et al. 2000), but no formal assessment has been done since.

\newpage{}

\section{Data}\label{data}

Many sources of data were available for this assessment, including
indices of abundance (Table 5), length observations, and age
observations from fishery-dependent and fishery-independent sources.

\subsection{Fishery-independent data}\label{fishery-independent-data}

Data from three fishery-independent surveys were used in this
assessment: 1) the SWFSC and NWFSC/PWCC Midwater Trawl Survey
(hereafter, ``juvenile survey''); 2) the Alaska Fisheries Science Center
(AFSC)/NWFSC Triennial Shelf Trawl Survey (hereafter, ``triennial
survey''); and 3) the NWFSC West Coast Groundfish Bottom Trawl Survey
(hereafter, ``WCGBTS''). These surveys employed different designs and
sampling methodologies, were conducted during different years and time
periods within years, and included coverage over different areas of the
coast. In some instances, the survey frequency, depths, and geographic
areas covered were not internally consistent within surveys. A brief
description of each survey is provided below.

Strata were defined by latitude and depth to analyze the catch-rates,
length compositions, and age compositions using stratified random
sampling theory (Table 6 \& Table 7). The latitude and depth breaks were
chosen based on the design of the survey as well as by looking at
biological patterns in relation to latitude and depth. Indices of
abundance for all of the surveys were derived using model based
approaches described below.

\subsubsection{\texorpdfstring{\acrlong{s-juv}}{}}\label{section}

We updated the coastwide pre-recruit index of abundance for widow
rockfish using data from three midwater trawl surveys targeting
young-of-the-year (YOY) rockfish (\gls{s-juv}), provided by Tanya Rogers
(SWFSC, pers. comm.). All surveys used identical gear, enabling the
construction of a consistent coastwide index spanning from 36°N to the
U.S./Canada border since 2004. For building the widow rockfish
pre-recruit index, we used data from 2001 to 2024 without spatial
subsetting (including CA, OR, and WA). Sampling in 2020 was limited due
to the COVID-19 pandemic and excluded from all models. In 2010 and 2012,
coverage was incomplete, so these years were used to construct the index
but excluded from the final model to align with the 2019 assessment.
Data from 2001--2003 were also excluded following the 2015 assessment
due to limited spatial coverage (36°30′ to 38°20′ N latitude). However,
a sensitivity analysis was conducted to examine the impact of including
those early years (see Figure: Sensitivity Analysis).

The index was built using a spatial GLM with the sdmTMB package
{[}anderson\_sdmtmb\_2022{]}, modeling 100-day standardized
catch-per-tow as a function of year (fixed effect), Julian date (GAM
smoother, k = 4), spatial random field, and spatiotemporal random
effects. Models with Tweedie, delta-lognormal, and delta-gamma error
structures were compared; DHARMa residuals and simulation-based
diagnostics indicated the Tweedie model performed best. The index shows
a strong increasing trend in juvenile abundance from 2017 to 2023, with
a slight decline in 2024. Despite the dip, recent values remain high
relative to the previous decade, and uncertainty estimates support the
robustness of this trend.

\subsubsection{\texorpdfstring{\acrlong{s-tri}}{}}\label{section-1}

The \gls{s-tri} was first conducted by the \gls{afsc} in 1977, and the
survey continued until 2004 (\textbf{weinberg\_2001\_2002?}). Its basic
design was a series of equally-spaced east-to-west transects across the
continential shelf from which searches for tows in a specific depth
range were initiated. The survey design changed slightly over time. In
general, all of the surveys were conducted in the mid summer through
early fall. The 1977 survey was conducted from early July through late
September. The surveys from 1980 through 1989 were conducted from the
middle of July to late September. The 1992 survey was conducted from the
middle of July through early October. The 1995 survey was conducted from
early June through late August. The 1998 survey was conducted from early
June through early August. Finally, the 2001 and 2004 surveys were
conducted from May to July.

Haul depths ranged from 91--457 m during the 1977 survey. Due to haul
performance issues and truncated sampling with respect to depth, the
data from 1977 were omitted from this analysis. The surveys in 1980,
1983, and 1986 covered the U.S. West Coast south to 36.8\textdegree N
latitude and a depth range of 55--366 m. The surveys in 1989 and 1992
covered the same depth range but extended the southern range to
34.5\textdegree N (near Point Conception). From 1995 through 2004, the
surveys covered the depth range 55--500 m and surveyed south to
34.5\textdegree N. In 2004, the final year of the \gls{s-tri} series,
\gls{nwfsc} \gls{fram} conducted the survey following similar protocols
to earlier years.

\subsubsection{\texorpdfstring{\acrlong{s-wcgbt}}{}}\label{section-2}

The \gls{s-wcgbt} is based on a random-grid design; covering the coastal
waters from a depth of 55--1,280 m (Bradburn, Keller, and Horness 2011).
This design generally uses four industry-chartered vessels per year
assigned to a roughly equal number of randomly selected grid cells and
divided into two `passes' of the coast. Two vessels fish from north to
south during each pass between late May to early October. There were
only two vessels used in 2019 and three in 2013, with one of the three
that year unable to complete its survey pass due to a government
shutdown. No survey occurred in 2020 due to \gls{covid}. This design
therefore incorporates both vessel-to-vessel differences in
catchability, as well as variance associated with selecting a relatively
small number (approximately 700) of possible cells from a very large set
of possible cells spread from the Mexican to the Canadian borders. Note
that the Survey is not permitted to access the \glspl{cca} in Southern
California.

Widow rockfish are not commonly caught in the WCGBTS. Higher catch rates
occur north of 40° N latitude and catches are rare south of 36° N
latitude (Figure 11). Few large fish are found shallower than 100 m and
few small fish are found in the deeper water of the slope. There is no
clear trend in length with latitude other than smaller fish tend to
occur south of approximately 36° N latitude, and there appears to be
some very small fish found near 39° N latitude.

Geostatistical models of biomass density were fit to survey data using
the R package \gls{sdmtmb} (\textbf{Anderson:2022:SRP?}). This approach
reflects an updated approach compared to the 2015 assessment
(non-spatial delta-GLMM) and the 2019 update assessment (VAST
delta-lognormal model). These models can account for latent spatial
factors with a constant spatial Gaussian random field and spatiotemporal
deviations to evolve as a random walk Guassian random field using a 200
knot grid of the survey area (\textbf{thorson\_geostatistical\_2015?}).
The prediction grid was also truncated to only include available survey
locations in depths between 55--500 m to limit extrapolating beyond the
data and edge effects. Tweedie, delta-binomial, delta-gamma, and mixture
distributions, which allow for extreme catch events, were investigated.
The positive catch weight model includes survey pass (`first' or
`second') to account for missing data, i.e., the incomplete second pass
of the 2013 survey. Vessel-year effects, which have traditionally been
included in index standardization for this survey, were not included as
the estimated variance for the random effect was close to zero.
Vessel-year effects were more prominent when models did not include
spatial effects and were included for each unique combination of vessel
and year in the data to account for the random selection of commercial
vessels used during sampling (\textbf{helser\_generalized\_2004?};
\textbf{thorson\_accounting\_2014?}).

Results are only shown for both the delta-gamma and delta-lognormal
distributions. Both models converged (positive, definite Hessian matrix)
but predicted data from both models showed slightly right-heavy tails
compared to null expectations, with the gamma model having stronger
divergence. The delta-lognormal distribution ulitmately had to the best
model diagnostics, e.g., similar distributions of theoretical normal
quantiles and model quantiles, high precision, lack of extreme
predictions that are incompatible with the life history, and low
\gls{aic}. Spatiotemporal estimates of positive catch for the
delta-lognormal distribution were then converted into annual indices
using sdmtmb::get\_index() function. (\textbf{Anderson:2022:SRP?})

The index estimate is relatively stable, with a slightly increasing
trend in recent years and a moderate peak in 2016. Overall, the
lognormal index estimates were more comparable to the 2019
spatiotemporal VAST-based index and seemed less influenced by potential
extreme catch events, particularly in 2013 and 2016; for these reasons,
in addition to better model performance described above, the
delta-lognormal sdmTMB-based index was used for the base model in this
assessment. The delta-lognormal mean value (2262.824) was slightly lower
than the means of the index values used in the 2015 assessment (2701.12)
and the 2019 update assessment (3301.765). Comparisons of the different
error structures, design-based estimate and the VAST index used in 2019
are in Figure XX.

\subsection{Fishery-dependent data}\label{fishery-dependent-data}

\subsubsection{Landings}\label{landings}

Widow rockfish have been caught in trawl and hook-and-line fisheries
since the early part of the 20th century. Widow rockfish are a desirable
rockfish and are not likely to be discarded for market reasons. However,
smaller widow rockfish are found at shallower depths and discarding
practices in the early 1900s are uncertain. In data from the early
1980s, widow rockfish have had their own landing category, beginning in
California in 1982, Oregon in 1984, and Washington in 1988. Estimates of
historical landings of widow rockfish rely upon species-composition
sampling data from each period. The uncertainty in species composition
is greater in past years, with less systematic and extensive sampling
occurring prior to 1980. Consequently, the precision with which landings
of widow rockfish can be estimated likely decreases for earlier years.

The definitions of fishing fleets have not been changed from those in
the 2015 and 2019 assessments. Five fishing fleets were specified within
the model: 1) a shorebased bottom trawl fleet with coastwide catches
from 1916--2024, 2) a shorebased midwater trawl fleet with coastwide
catches from 1979--2024, 3) a mostly midwater trawl fleet that targets
Pacific Hake/Whiting (Merluccius productus) and includes a foreign and
at-sea fleet with catches from 1975--2024, a domestic shorebased fleet
that targeted Pacific Hake with catches from 1991--2024, and foreign
vessels that targeted Pacific Hake and rockfish between 1966--1976, 4) a
net fishery consisting of catches mostly from California from
1981--2024, and 5) a hook-and-line fishery (predominantly longline) with
coastwide catches from 1916--2024. As in previous assessments, catches
from Puget Sound and those from commercial shrimp trawls, commercial
pots, and recreational fisheries were excluded (as these are generally
minimal).

Catches from all years (1916-2018) were carried forward into this
assessment, with two exceptions. First, discards from the hook-and-line
fleet were added to the removals for this fleet . The hook-and-line
removals of widow rockfish are extremely minimal (Figure 1) and
comprised only approximately 0.2\% of the total removals over the last
twenty years, with discard being a small fraction of that. The
biological samples of the discard amount are also scarce, with input
sample sizes not exceeding 6 and averaging around 3 per year (Table X).
With this limited data, the model was unable to reliably estimate
retention parameters and exhibited substantial sensitivity to even
slight changes in discard amounts within the hook-and-line fleet.
Therefore, in this assessment, we added hook-and-line discards to
hook-and-line landings. Second, because PacFIN appear to underestimate
midwater trawl catches in California in 1979-1980 (Edward Dick, Pers.
Comms.) we adjusted midwater and bottom trawl catches from California in
these years to reflect the ratio of California midwater to bottom trawl
catches in 1981-1982. New catches (2019-onward) from PacFIN and ASHOP
were otherwise appended onto the 1916-2018 landings and apportioned
among fleets using the same criteria as those documented in the 2015
assessment.

\subsubsection{Fishery length and age
data}\label{fishery-length-and-age-data}

Biological data from commercial fisheries that caught widow rockfish
were extracted from PacFIN (PSMFC) on July 3, 2019, from CALCOM on July
3, 2019 and from the NORPAC database on July 3, 2019. Lengths taken
during port sampling in California, Oregon, and Washington were used to
calculate length and age compositions. The data were classified into
bottom trawl, midwater trawl, hake trawl, net, and hook-and-line fleets

Table 10 shows the number of landings sampled and Table 11 shows the
number of lengths taken for each year, gear, and fleet from the three
states. Table 12 shows these numbers for the at-sea fleet.

Consistent with the 2015 assessment, length and age samples from PacFIN
and CALCOM were expanded up to the total landing then combined into
state-specific frequencies (Table 13). Expansion factors were calculated
in a way such that large expansions would not occur and based on ideas
first presented by Owen Hamel (pers. comm., NWFSC). First the expansion
factor (Ek) was the total catch weight (Wk) divided by the sample weight
(wk), and raised to 0.9 to account for non-homogeneity within a trip.
Then, expansion factors greater than 300 were capped (100 for net
fisheries) to reduce the influence of small samples (i.e., a few fish
representing a large catch). The predicted total numbers at length or
age weighted by landings for each state were added to create a
coast-wide length frequency. The effective sample sizes of the state
combined length frequencies were determined from the following formula,
which has been used in previous widow rockfish assessments as well as
other west coast groundfish assessments.

\begin{longtable}[]{@{}
  >{\raggedright\arraybackslash}p{(\columnwidth - 2\tabcolsep) * \real{0.4948}}
  >{\raggedright\arraybackslash}p{(\columnwidth - 2\tabcolsep) * \real{0.5052}}@{}}
\toprule\noalign{}
\begin{minipage}[b]{\linewidth}\raggedright
Fishery Samples
\end{minipage} & \begin{minipage}[b]{\linewidth}\raggedright
Survey Samples
\end{minipage} \\
\midrule\noalign{}
\endhead
\bottomrule\noalign{}
\endlastfoot
\(\ N_{eff} = N_{sample} + 0.138N_{fish} , \frac{N_{fish}}{N_{sample}} < 44\)
&
\(\ N_{eff} = N_{sample} + 0.0707N_{fish} , \frac{N_{fish}}{N_{sample}} < 55\) \\
\(\ N_{eff} = 7.06N_{sample} , \frac{N_{fish}}{N_{sample}} \geq 44\) &
\(\ N_{eff} = 4.89N_{sample} , \frac{N_{fish}}{N_{sample}} \geq 55\) \\
\end{longtable}

This is slightly different than the sample size of 2.43 per haul for
rockfish that (\textbf{stewart\_bootstrapping\_2014?}) report. Observed
lengths were expanded to the tow from At-Sea Hake Observer Program
samples (NORPAC). Tows are typically well sampled, thus expansion
factors were not modified from what was calculated. Hake fishery length
compositions were created by combining shoreside and at-sea length
compositions, weighting by the catch from each sector. The effective
sample sizes for hake fishery length and age comps were calculated using
the above equations for the shoreside fleet and added to the number of
tows sampled from the at-sea fleet.

Expanded length compositions for bottom trawl, midwater trawl, hake
fisheries, net, and hook-and-line are shown in Figure 17 to Figure 21.
It is quickly apparent that all of these fisheries rarely land fish less
than 26 cm. All of the non-hake fleets show a strong cohort coming
though in the late 1970s and early 1980s, and then another cohort coming
through in the late 1980s. Sample sizes typically dropped off after
2000, except in the hake fishery where nearly every tow is sampled. Age
compositions for the five fleets are shown in Figure 22 and Figure 26.
Occasional cohorts appear to move through the population, indicating
that widow rockfish population dynamics may be characterized by episodic
recruitment events.

\subsubsection{Discards}\label{discards}

-- TO DO!!! ---

\[
D_{y,f} = \frac{d_{y,f}}{r_{y,f}} R_{y,f}
\]

\subsubsection{Biological data}\label{biological-data}

\paragraph{Weight-length relationship}\label{weight-length-relationship}

Weight-at-length data, which are the same used in the 2015 assessment,
were collected from fisheries sampling and by the Triennial and NWFSC
WCGBT Surveys, and were used to estimate a weight-length relationship
for widow rockfish (Figure 30). Weight-at-length was similar between
sources with the fishery samples showing a slightly smaller weight at
large sizes when compared to the survey data (Figure 31). WCGOP data
were not used because only small fish were sampled, the weight of these
small fish were typically less than from other sources (Figure 30), and
the curves fitted to only WCGOP data were unable to estimate the slope.
There were only 81 observations from the WCGOP data, which is a small
amount of data compared to everything available. However, these
observations may be useful to understand discards.

The weight-length relationship used in the 2011 assessment was similar
for males but predicted slightly heavier females at larger sizes than
the 2015 assessment (Figure 31). The following relationships between
weight and length for females and males were estimated for the 2015
assessment from all of the data combined and were used in the current
assessment:

\[
\text{Females: } \qquad weight = 1.7355 \times 10^{-5} \cdot Length^{2.9617}
\]

\[
\text{Males: } \qquad weight = 1.4824 \times 10^{-5} \cdot Length^{3.0047}
\]

where weight is measured in kilograms and length in cm. These
relationships were used in the assessment as fixed relationships.

\paragraph{Maturity schedule}\label{maturity-schedule}

Estimates of maturity used in this update were the same as the 2015
assessment. Estimates of maturity at length have been presented by
(\textbf{barss\_maturity\_1987?}),
(\textbf{echeverria\_thirty-four\_1987?}), and
(\textbf{love\_life\_1990?}). (\textbf{barss\_maturity\_1987?}) supplied
data collected from Oregon and California commercial and recreational
samples, which allowed us to estimate the proportion mature-at-length
and proportion mature-at-age for samples from each state (Figure 32). As
noted by (\textbf{barss\_maturity\_1987?}), the samples from Oregon
matured at older age and larger length. Estimates of maturity-at-length
from California reported by (\textbf{barss\_maturity\_1987?}) are
similar to estimates of length-at-50\%-mature from samples collected in
California reported by (\textbf{echeverria\_thirty-four\_1987?}) and
(\textbf{love\_life\_1990?}), although (\textbf{barss\_maturity\_1987?})
show the smallest length-at-50\%-mature. To maintain some consistency
with the 2011 assessment and to avoid any potential growth issues by
area, the 2015 assessment used maturity-at-age data from the 2011
assessment, but used the data provided by
(\textbf{barss\_maturity\_1987?}) to estimate a new maturity curve
following a logistic function with the data from California and Oregon
equally weighted to avoid California dominating the estimated
relationship. This maturity-at-age curve falls between the estimated
California and Oregon maturity-at-age curves (Figure 32, right), with
the age-at-50\%-mature estimated at 5.47 and with a slope of -0.7747 (as
specified in SS). This logistic maturity-at-age curve was used in the
2015 and 2019 update assessment except that maturity-at-age for ages 2
and lower were set equal to zero (Table 19).

\paragraph{Fecundity}\label{fecundity}

Fecundity in rockfish is often not a linear function of weight, but
increases faster at larger weights (Dick 2009). Therefore, this
relationship is often accounted for in rockfish assessments by using
spawning output (numbers of eggs) to determine current status. (Dick
2009) did not find a significant relationship between the number of eggs
per gram of body weight and body weight for widow rockfish. Therefore,
spawning output was assumed to be proportional to weight, which is the
same as spawning biomass, and is reported here.

\paragraph{Natural Mortality}\label{natural-mortality}

Natural mortality used in this update differed from the 2015 assessment.
Natural mortality (M) is a parameter that is often highly uncertain in
fish stocks. Past assessments of widow rockfish assumed constant natural
mortality of 0.125 yr-1 or 0.15 yr-1. The 2011 assessment estimated M
with a prior developed by Owen Hamel (NWFSC, pers. comm.) using methods
described in (Hamel 2014). This prior was based on a maximum age of 44
and 40 for females and males, respectively, a mean temperature of 8
degrees Celsius (about 150m deep off of Oregon), and a gonadosomatic
index of 9.99\% and 1.86\% for females and males, respectively
(\textbf{love\_life\_1990?}). The sex-specific lognormal priors for M
have medians of 0.124 yr-1 and 0.129 yr-1 for females and males,
respectively, and a coefficient of variation (CV) of 30.7\% for each
sex. In 2015, discussions with Owen Hamel (NWFSC) led to the development
of a new prior based solely on maximum age to use when estimating M.
Using all of the available age data, a maximum age of 54 was determined
for both females and males, although it has been rare to observe widow
rockfish older than about 45 years old (Figure 33). This resulted in a
prior with a much smaller median (0.0810 or -2.513284 in log space) and
a larger standard deviation in log space (0.523694). For the update
assessment, an updated meta-anaysis resulted in a prior with a slightly
smaller median than the 2015 assessment (0.10 or -2.30 in log space) and
a smaller standard deviation in log space (0.438). Figure 34 shows that
these prior distributions are wide and not highly informative.

\paragraph{Length-at-age}\label{length-at-age}

Estimates of length-at-age used in this update were the same as the 2015
assessment. Two different labs have aged the majority of processed
otoliths for widow rockfish. The SWFSC has been aging widow rockfish
otoliths for many years, including all of the fishery data prior to 2011
and otoliths collected from the NWFSC WCGBT survey in 2009 and 2010. The
Cooperative Ageing Project (CAP) in Newport, Oregon aged 1,100 otoliths
from the NWFSC WCGBT survey, 2,026 otoliths provided by ASHOP, and 3,467
otoliths collected by port samplers. All of the commercial fishery
samples were collected in the years 2011--2014. In total, there are
105,814 paired age and length observations ranging from 1978 to 2014.
Figure 35 shows the lengths and ages for all years and all data as well
as predicted von Bertalanffy fits to the data. Females grow larger than
males and sex specific growth parameters were estimated at the following
values:

\[
\text{Females: } \qquad  L_\infty = 50.34,\quad k=0.15, \quad t_0 = -2.22
\]

\[
\text{Males: } \qquad  L_\infty = 44.19,\quad k=0.21, \quad t_0 = -1.78
\]

The data from each source (ASHOP, port sampling/BDS, Triennial survey,
and NWFSC survey) are shown in Figure 36 with fitted von Bertalanffy
lines. All of these sources are quite similar, especially observations
from ASHOP and the NWFSC survey. The standard deviation (SD) and
coefficient of variation (CV) of length-at-age are shown in Figure 37.
Modelling the CV as a function of predicted length-at-age appears to be
somewhat linear from a value just over 0.1 at small lengths and slightly
less than 0.045 at larger lengths. However, variance in length- at-age
was estimated separately in stock-synthesis.

\paragraph{Sex ratios}\label{sex-ratios}

Females tend to grow larger than males and it is expected that the
proportion of females approaches one at large lengths and is less than
0.5 at intermediate lengths. Figure 38 shows that the proportion of
females at length from survey data is approximately 50\% until
approximately 34 cm, when the proportion of females drops below 50\%. At
lengths larger than 46 cm, the proportion of females increases rapidly
to one, suggesting that few males grow larger than 50 cm.

\paragraph{Ageing bias and
imprecision}\label{ageing-bias-and-imprecision}

Uncertainty surrounding the ageing-error process for widow rockfish used
in the 2015 assessment was incorporated by estimating ageing error by
age. No changes were made from the 2015 assessment for the update.
Age-composition data used in the model were from break-and-burn and
surface reads and were aged by the Cooperative Ageing Project (CAP) in
Newport, Oregon and the SWFSC in Santa Cruz, California. 12
Break-and-burn double reads of 1788 otoliths were performed by both the
CAP and the SWFSC lab combined. Additionally, 100 otoliths were read
both by surface and break-and-burn methods. An ageing error estimate was
made based on these double reads using a computational tool specifically
developed for estimating ageing error (Punt et al. 2008), and using
release 1.0.0 of the R package nwfscAgeingError
(\textbf{thorson\_nwfscageingerror\_2012?}) for input and output
diagnostics, publicly available at: https://github.com/nwfsc-
assess/nwfscAgeingError. The maximum aged fish read by the surface
reading method was 10 years and the cross otolith reads between the
surface and break-and-burn ageing methods showed limited variation.
Therefore, a unique ageing error was not created for surface read
otoliths. A non-linear standard error was estimated by age where there
is more variability in the estimated age of older fish was estimated for
each reading lab (Table 20 and Figure 39).

\subsubsection{Abundance indices}\label{abundance-indices}

\subsection{Environmental and ecosystem
data}\label{environmental-and-ecosystem-data}

\newpage{}

Assessment model

An age-structured stock assessment model was used to predict the biomass
trajectory of Widow Rockfish with an approach of balancing parsimony
with complexity. This allowed for the determination of general trends in
the biomass over time without introducing extraneous data partitions
that explain little additional variation. The assessment followed the
same model structure as the 2015 base assessment.

\subsection{History of modeling
approaches}\label{history-of-modeling-approaches}

Refer to the most recent full assessment for additional information.

\subsection{Responses to SSC Groundfish Subcommittee
requests}\label{responses-to-ssc-groundfish-subcommittee-requests}

To be completed after review.

\subsection{Model Structure and
Assumptions}\label{model-structure-and-assumptions}

For this update assessment, new versions of the previously used software
were used. Stock Synthesis v3.30.13 was used to estimate the parameters
in the 2019 model. R4SS, version 1.35.3, along with R version 3.5.3 were
used to investigate and plot the 2019 model fits. For the update, Stock
Synthesis v3.30.2 and R4SS, version 1.51.0 , along with R version 4.4.2
were used. Bridging from Stock Synthesis v3.24U to v3.30.13 is
illustrated in Figure 41. A summary of the data sources used in the
model (details discussed above) is shown in Figure 40. Stock Synthesis
has many options when setting up a model and the assessment model for
Widow Rockfish was set up in the following manner.

\subsubsection{Model Changes from the Last
Assessment}\label{model-changes-from-the-last-assessment}

\subsubsection{Modeling Platform and
Structure}\label{modeling-platform-and-structure}

\subsubsection{Model Parameters}\label{model-parameters}

\subsubsection{Key Assumptions and Structural
Choices}\label{key-assumptions-and-structural-choices}

Refer to the most recent full assessment for additional information.

\subsection{Base Model Results}\label{base-model-results}

\subsubsection{Parameter Estimates}\label{parameter-estimates}

3.3.1 Parameter estimates

The estimates of natural mortality 0.124599 yr-1 and NA yr-1 for females
and males, respectively) were higher than suggested by the medians of
the prior distributions used in this assessment and the 2015 assessment.
Fixing M at lower values than those estimates resulted in a pattern of
reduced recruitment immediately before the fishery started. This
suggests that the model is doing what it can to reduce the number of
observations of older fish in the data. The estimates of M fall within
the 95\% confidence interval of the prior distribution (0.0425--0.237),
and are shown in Figure 44.

Estimating M is difficult in stock assessments, and the estimated values
may represent model misspecification instead of the actual life-history
trait. However, in alternative models to the base case model, the
estimates of M were rarely less than 0.14 yr-1 (Table 29). Uncertainty
in the estimated M was also much less than the range of the prior
(Figure 44). The assumption that appeared to have the largest effect on
M was introducing dome-shaped selectivity in the midwater trawl fleet,
which made M smaller (Table 29).

Selectivity curves were estimated for commercial and survey fleets and
parameter estimates are provided in Table 24. The final base model
assumed asymptotic selectivity (double-normal selectivity curve ) for
each fishery, except for the midwater trawl fishery. The NWFSC and
Triennial surveys both used spline curves. All selectivity curves were
length-based and are the same shape as in the 2015 benchmark. Time
blocks were used for the bottom trawl, midwater trawl and hook-and-line
fisheries as indicated in Table 21. The estimated selectivity,
retention, and keep (the product of selectivity and retention) curves
for the trawl and hook-and-line fleets are shown in Figure 45. The
selectivity curves showed a shift to larger fish in 2002 for the bottom
trawl fishery and a shift to smaller fish in 2003 for the hook-and-line
fishery. The bottom trawl shift is consistent with the introduction of
the RCA and gear restrictions (shoreward of the 18 RCA) that virtually
eliminated fishing in shelf habitats where smaller Widow Rockfish would
more likely be encountered. Around this same time, the fixed-gear RCA
specifications began preventing fishing between 30 and 100 fm.

The retention curves showed a shift to retaining a lower percentage of
fish since trip limits were introduced, but increases in recent years.
The asymptote of the retention curve for the bottom trawl fishery
sequentially decreased as more management restrictions were introduced
to about 50\% retention of larger fish in the 1998-2010 period.

Midwater trawl and hook-and-line fisheries estimated an asymptote to
retention just above 80\% for the period 1983-2010. Both the selectivity
for the hake fleet and the selectivity of the net fleet did not support
dome-shaped selectivity (Figure 46). The estimated selectivity curves
for the Triennial and NWFSC WCGBT surveys were similar to each other
except that the triennial survey selected larger fish (Figure 46). The
NWFSC WCGBT survey was no longer minimally dome-shaped as in the 2015
assessment.

In 2015, additional survey variability (process error added directly to
each year's input variability) for the triennial and NWFSC WCGBT surveys
was not estimated in the model because the estimate was zero. To avoid
bound issues in estimation of the Hessian, the authors fixed these at
zero because the model- based results provided reasonable estimates of
variance. We retained the same modelling approach for the update
assessment. The additional standard deviation added to the
fishery-dependent indices was quite large, ranging from 0.16 for the
bottom trawl index and 0.58 for the foreign at-sea hake fleet. The
additional variability on the juvenile survey was the highest, at 0.83,
giving the index very little weight in the model.

The estimates of maximum size for both females and males (Table 23) were
not unexpected given the data in Figure 35. Estimates of k were slightly
different in the model, but that is expected when accounting for
selectivity. Estimated growth curves are shown in (Figure 47).

Estimates of recruitment suggest that the Widow Rockfish population is
characterized by variable recruitment with occasional strong
recruitments and periods of low recruitment (Figure 48). There is little
information regarding recruitment prior to 1965, and the uncertainty in
these estimates is expressed in the model. There are very large, but
uncertain, estimates of recruitment in 2013, 1970, 2008, and 1971. Other
large recruitment events (in descending order of magnitude) occurred in
1978, 2014, 1981, 2010, and 1991. The five lowest recruitments (in
ascending order) occurred in 2012, 2011, 1976, 2007, and 1973. Estimates
of recruitment appear to be episodic and characterized by periods of low
recruitment. Two of the four largest estimated recruitments occurred in
the last 11 years.

\subsubsection{Fits to the Data}\label{fits-to-the-data}

\subsubsection{Population Trajectory}\label{population-trajectory}

\subsection{Model Diagnostics}\label{model-diagnostics}

Three types of uncertainty are presented for the assessment of Widow
Rockfish. First, uncertainty in the parameter estimates was determined
using approximate asymptotic estimates of the standard error. These
estimates were based on the maximum likelihood theory that the inverse
of the Hessian matrix (the second derivative of the log-likelihood
function with respect to the parameter vector) approaches the true
uncertainty of the parameter estimates as the sample size approaches
infinity. This approach takes into account the uncertainty in the data
and supplies correlation estimates between parameters, but does not
capture possible skewness in the error distribution of the parameters
and may not accurately estimate the standard error in some cases (see
Stewart et al.~2013) UPDATE REF!!!!!!.

The second type of uncertainty that is presented is related to modeling
and structural error. This uncertainty cannot be captured in the base
model as it is related to errors in the assumptions used in specifying
the base model. Therefore, sensitivity analyses were conducted where
assumptions were modified to reveal the effect they have on the model
results.

Lastly, a major axis of uncertainty was determined from a parameter or
structural assumption that results in the greatest change in stock
status and advice, and projections were made for different states of
nature based upon that parameter or structural assumption.

\subsubsection{Convergence}\label{convergence}

\subsubsection{Parameter Uncertainty}\label{parameter-uncertainty}

Parameter estimates are shown in Table 22, Table 23, and Table 28 along
with approximate asymptotic standard errors. The only parameters with an
absolute value of correlation greater than 0.95 were the female and male
natural mortality parameters, which is expected. Estimates of key
derived quantities are given in Table 26 along with approximate 95\%
asymptotic confidence intervals. There is a reasonable amount of
uncertainty in the estimates of biomass. The confidence interval of the
2019 estimate of depletion is XX\%--XX\% and above the management target
of 40\% of the unfished spawning biomass.

\subsubsection{Sensitivity Analyses}\label{sensitivity-analyses}

Sensitivity analysis was performed to determine the model behavior under
different assumptions than those of the base case model. Seven
sensitivity analyses were conducted to explore the potential differences
in model structure and assumptions, including:

\begin{enumerate}
\def\labelenumi{\arabic{enumi}.}
\tightlist
\item
  Fixed natural mortality at 0.1 for both sexes (2015 assessment prior)
\item
  Fixed natural mortality at 0.124 yr-1 for females and 0.129 yr-1 for
  males (2011 assessment prior)
\item
  Fixed steepness at 0.4
\item
  Fixed steepness at 0.6
\item
  Fixed steepness at 0.798 (2015 assessment value)
\item
  Forcing asymptotic selectivity on the midwater trawl fleet
\item
  Fitting logistic curves for NWFSC WCGBT survey selectivities
\item
  Weighting the composition data using the Francis method
\item
  Updated Washington catch reconstruction
\item
  Inclusion of previously excluded shrimp trawl data
\item
  Exclusion of triennial survey data
\end{enumerate}

Likelihood values and estimates of key parameters are shown in Table 29.
Predicted spawning biomass trajectories and estimated recruitment
deviations are shown in in Figure 65. The estimates of current stock
depletion ranged from 55\%-142\% across the sensitivity runs, with
fixing natural mortality at 0.1 resulting in the lowest estimate and
forcing asymptotic selectivity on the midwater trawl fleet resulting in
the highest estimate. Generally, the trajectory of the spawning biomass
was qualitatively similar across all tested models, e.g., peak around
late 1970s and late 2010s, projected decrease in biomass in 2025
followed by some recovery into the 2030s; the quantitative magnitude of
these trends did vary across cases.

Fixing M at values lower than the base case estimate resulted in
decreases in estimated spawning biomass, while fixing steepness across
the values tested resulted in similar or increased estimated spawning
biomass. The relative spawning biomass in 2025 changed to 93\% with an M
of 0.124 yr-1 and 0.129 yr-1 for females and males, respectively, and to
55\% with an M of 0.1 yr-1. Fixing steepness at a value of 0.6 resulted
in an increase of the spawning biomass to 125\% and a decrease in
equilibrium yield at a SPR50\% reference harvest rate, while other
tested values for fixed steepness had relatively minimal effects on
spawning biomass and equilibrium yield at a SPR50\% reference harvest
rate. Fixing steepness at a value of 0.4 resulted in low recruitment
deviations in the 2019-2024 period relative to other tested models and
the base model.

Forcing asymptotic selectivity on the midwater fleet resulted in the
largest estimated spawning biomass for 2025, while forcing logistic
selectivity on the NWFSC WCGBT resulted in similar estimated spawning
biomass to the base model. Including shrimp trawl data and updating WA
catch reconstruction had almost no impact on the estimated spawning
biomass. Excluding the triennial survey data lead to slight increases in
estimated spawning biomass.

The alternative weighting using the Francis method generally increased
the estimate of spawning biomass across the timeseries, but the
estimated biomass for 2025 was similar between the Francis weighted
model and the base model.

\subsubsection{Retrospective Analysis}\label{retrospective-analysis}

A 5-year retrospective analysis was conducted by running the model using
data only through 2020, 2021, 2022, 2023, and 2024 progressively (Table
30 and Figure 66). The initial scale of the spawning population was
basically unchanged for all of these retrospectives. Removing 4--5 years
of data led to slightly lower estimates of fishing mortality (F) and
slightly higher spawning biomass over the last 15 years. In contrast,
removing only 1--2 years resulted in higher F and lower biomass
estimates. Despite these minor differences, all retrospectives showed a
consistent declining trend in spawning biomass over the past decade. No
concerning patterns were observed in the retrospective analysis.

\subsubsection{Likelihood Profiles and key
parameters}\label{likelihood-profiles-and-key-parameters}

Likelihood profiles were conducted for R0, steepness (even though it was
not estimated in the base case) and over male and female natural
mortality values simultaneously. These likelihood profiles were
conducted by fixing the parameter at specific values and removing the
prior on the parameter being profiled. Without the original prior
distribution the MLE estimates from the base case will likely be
different than the MLE in the likelihood profile, but this displays what
information the data have. There was some difficulty in achieving model
convergence for many parameterizations in the likelihood profile. In
some cases jittering was required.

As R0 increased, natural mortality also increased and the relative
spawning biomass in 2015 was less depleted (Table 31). There was
variable support for each likelihood component across the range of R0
evaluated. The total likelihood supported the estimated value (Table
31). Profiles are illustrated in Figure 68.

\subsection{Unresolved Problems and Major
Uncertainties}\label{unresolved-problems-and-major-uncertainties}

\newpage{}

\section{Management}\label{management}

\subsection{Reference Points}\label{reference-points-1}

catches (landings plus discards) have been below the point estimate of
potential long-term yields calculated using an SPR50\% reference point
and the population has been increasing over the last decade. However,
catches in 2018 were above the point estimate of potential long-term
yields calculated using an SPR50 reference point.

The predicted spawning biomass from the base model generally showed a
slight decline until the late 1970s, steep increase above unfished
equilibrium levels, then a steep decline until the mid-1980s followed by
less of a decline until 2001 (Figure 61). Since 2001, the spawning
biomass has been increasing due to small catches, and recently, above
average recruitment. The 2018 spawning biomass relative to unfished
equilibrium spawning biomass is above the target of 40\% of unfished
spawning biomass (Figure 63). The fishing intensity (relative 1-SPR)
exceeded the current estimates of the harvest rate limit (SPR50\%)
throughout the 1980s and early 1990s, as seen in Figure 73. Recent
exploitation rates on Widow Rockfish were predicted to be much less than
target levels. In recent years, the stock has experienced exploitation
rates that have been below the target level while the biomass level has
remained above the target level (Figure 74).

The equilibrium yield plot is shown in Figure 75, based on a steepness
value fixed at 0.720. The predicted maximum sustainable yield under the
assumptions of this assessment occurs near 25\% of equilibrium unfished
spawning biomass.

\subsection{Unresolved problems and major
uncertainties}\label{unresolved-problems-and-major-uncertainties-1}

\subsection{Harvest Projections and Decision
Tables}\label{harvest-projections-and-decision-tables}

\begin{Shaded}
\begin{Highlighting}[]
\FunctionTok{load}\NormalTok{(}\AttributeTok{file=}\NormalTok{here}\SpecialCharTok{::}\FunctionTok{here}\NormalTok{(}\StringTok{"report"}\NormalTok{,}\StringTok{"tables"}\NormalTok{,}\StringTok{"projections.rda"}\NormalTok{))}
\NormalTok{projections  }\OtherTok{\textless{}{-}} \FunctionTok{as.data.frame}\NormalTok{(projections}\SpecialCharTok{$}\NormalTok{table)}


\NormalTok{projections }\OtherTok{\textless{}{-}}\NormalTok{ projections}\SpecialCharTok{|\textgreater{}}\NormalTok{dplyr}\SpecialCharTok{::}\FunctionTok{mutate}\NormalTok{(}\FunctionTok{across}\NormalTok{(}\DecValTok{2}\SpecialCharTok{:}\FunctionTok{ncol}\NormalTok{(projections), }\SpecialCharTok{\textasciitilde{}} \FunctionTok{round}\NormalTok{(.x)))}
\end{Highlighting}
\end{Shaded}

Projections with catches based on the predicted annual catch limit (ACL)
using the SPR rate of 50\%, the 40:10 control rule, and a 0.25 P*
adjustment using a sigma of 0.50 from 2021 onward suggest that the
spawning biomass will decrease over the projection period for all states
of nature. Predicted ACL catches range from 10,961 mt in 2021 to 5,944
mt in 2030.

\textbf{\emph{nOT SURE WHERE acl CATCHES ARE }}

\subsection{Evaluation of Scientific
Uncertainty}\label{evaluation-of-scientific-uncertainty}

\subsection{Regional management
considerations}\label{regional-management-considerations}

Widow Rockfish have shown latitudinal differences in life-history
parameters, which has led past assessment authors to pursue a two-area
model. Modelling a stock with two areas is difficult because it requires
many assumptions about recruitment distribution, movement, and
connectivity, while also splitting data into two areas that reduces
sample sizes when compared to a coastwide model. The upside is that it
can result in a better model that more accurately predicts regional
status. This assessment is a coastwide model because not enough is known
about the assumptions that would have to be made for a two-area model.

It is still important to consider regional differences when making
management decisions. Following recent cohorts through time with survey
data showed that older fish showed up in the north after younger fish
were observed in the south (Figure 2). This may indicate connectivity
between the north and the south and that this is truly one stock.
However, more investigation is needed.

Widow Rockfish are managed on a coastwide basis and observed more often
in the NWFSC WCGBT bottom trawl survey north of latitude 40° 10′ N.
Bottom trawl catches in California have historically been as large as in
Oregon and larger than in Washington, but recently catches in California
have been small. Rockfish Conservation Areas (RCAs) cover a significant
proportion of Widow Rockfish habitat, but a midwater trawl fishery is
beginning to re-develop that can fish in these areas. Future assessments
and management of Widow Rockfish may want to monitor where catches are
being taken to make sure that specific areas are not being
overexploited. In addition, research on the connectivity along the coast
as well as regional differences would help to inform the potential for
overfishing specific areas.
\textgreater\textgreater\textgreater\textgreater\textgreater\textgreater\textgreater{}
1ed44512b0ba062144bae79bac8c50155d6b6fec

\subsection{Research and Data Needs}\label{research-and-data-needs-1}

\newpage{}

\subsection{Acknowledgements}\label{sec-acknowledgements}

\newpage{}

\subsection{References}\label{references}

\newpage{}

\section{Tables}\label{tables}

\subsection{Data}\label{data-1}

\subsubsection{Fishery-dependent data}\label{fishery-dependent-data-1}

:::

\newpage{}

::: \{\#tbl-hake\_removals .cell tbl-cap=' Landings (mt) from the
foreign \& domestic at-sea fleet and the domestic shoreside hake fleet.
Catches (mt) from the Pacific whiting at-sea fishery as determined by
onboard observers.'\} ::: \{.cell-output-display\}

\global\setlength{\Oldarrayrulewidth}{\arrayrulewidth}

\global\setlength{\Oldtabcolsep}{\tabcolsep}

\setlength{\tabcolsep}{2pt}

\renewcommand*{\arraystretch}{1.5}



\providecommand{\ascline}[3]{\noalign{\global\arrayrulewidth #1}\arrayrulecolor[HTML]{#2}\cline{#3}}

\begin{longtable*}[c]{|p{0.75in}|p{0.75in}|p{0.75in}|p{0.75in}|p{0.75in}}



\ascline{1.5pt}{666666}{1-5}

\multicolumn{1}{>{\raggedleft}m{\dimexpr 0.75in+0\tabcolsep}}{\textcolor[HTML]{000000}{\fontsize{11}{11}\selectfont{\global\setmainfont{Helvetica}{Year}}}} & \multicolumn{1}{>{\raggedright}m{\dimexpr 0.75in+0\tabcolsep}}{\textcolor[HTML]{000000}{\fontsize{11}{11}\selectfont{\global\setmainfont{Helvetica}{Foreign...Domestic}}}} & \multicolumn{1}{>{\raggedright}m{\dimexpr 0.75in+0\tabcolsep}}{\textcolor[HTML]{000000}{\fontsize{11}{11}\selectfont{\global\setmainfont{Helvetica}{Shoreside.hake}}}} & \multicolumn{1}{>{\raggedright}m{\dimexpr 0.75in+0\tabcolsep}}{\textcolor[HTML]{000000}{\fontsize{11}{11}\selectfont{\global\setmainfont{Helvetica}{X}}}} & \multicolumn{1}{>{\raggedright}m{\dimexpr 0.75in+0\tabcolsep}}{\textcolor[HTML]{000000}{\fontsize{11}{11}\selectfont{\global\setmainfont{Helvetica}{X.1}}}} \\

\ascline{1.5pt}{666666}{1-5}\endfirsthead 

\ascline{1.5pt}{666666}{1-5}

\multicolumn{1}{>{\raggedleft}m{\dimexpr 0.75in+0\tabcolsep}}{\textcolor[HTML]{000000}{\fontsize{11}{11}\selectfont{\global\setmainfont{Helvetica}{Year}}}} & \multicolumn{1}{>{\raggedright}m{\dimexpr 0.75in+0\tabcolsep}}{\textcolor[HTML]{000000}{\fontsize{11}{11}\selectfont{\global\setmainfont{Helvetica}{Foreign...Domestic}}}} & \multicolumn{1}{>{\raggedright}m{\dimexpr 0.75in+0\tabcolsep}}{\textcolor[HTML]{000000}{\fontsize{11}{11}\selectfont{\global\setmainfont{Helvetica}{Shoreside.hake}}}} & \multicolumn{1}{>{\raggedright}m{\dimexpr 0.75in+0\tabcolsep}}{\textcolor[HTML]{000000}{\fontsize{11}{11}\selectfont{\global\setmainfont{Helvetica}{X}}}} & \multicolumn{1}{>{\raggedright}m{\dimexpr 0.75in+0\tabcolsep}}{\textcolor[HTML]{000000}{\fontsize{11}{11}\selectfont{\global\setmainfont{Helvetica}{X.1}}}} \\

\ascline{1.5pt}{666666}{1-5}\endhead



\multicolumn{1}{>{\raggedleft}m{\dimexpr 0.75in+0\tabcolsep}}{\textcolor[HTML]{000000}{\fontsize{11}{11}\selectfont{\global\setmainfont{Helvetica}{}}}} & \multicolumn{1}{>{\raggedright}m{\dimexpr 0.75in+0\tabcolsep}}{\textcolor[HTML]{000000}{\fontsize{11}{11}\selectfont{\global\setmainfont{Helvetica}{At-sea}}}} & \multicolumn{1}{>{\raggedright}m{\dimexpr 0.75in+0\tabcolsep}}{\textcolor[HTML]{000000}{\fontsize{11}{11}\selectfont{\global\setmainfont{Helvetica}{CA}}}} & \multicolumn{1}{>{\raggedright}m{\dimexpr 0.75in+0\tabcolsep}}{\textcolor[HTML]{000000}{\fontsize{11}{11}\selectfont{\global\setmainfont{Helvetica}{OR}}}} & \multicolumn{1}{>{\raggedright}m{\dimexpr 0.75in+0\tabcolsep}}{\textcolor[HTML]{000000}{\fontsize{11}{11}\selectfont{\global\setmainfont{Helvetica}{WA}}}} \\





\multicolumn{1}{>{\raggedleft}m{\dimexpr 0.75in+0\tabcolsep}}{\textcolor[HTML]{000000}{\fontsize{11}{11}\selectfont{\global\setmainfont{Helvetica}{1,966}}}} & \multicolumn{1}{>{\raggedright}m{\dimexpr 0.75in+0\tabcolsep}}{\textcolor[HTML]{000000}{\fontsize{11}{11}\selectfont{\global\setmainfont{Helvetica}{3670}}}} & \multicolumn{1}{>{\raggedright}m{\dimexpr 0.75in+0\tabcolsep}}{\textcolor[HTML]{000000}{\fontsize{11}{11}\selectfont{\global\setmainfont{Helvetica}{0}}}} & \multicolumn{1}{>{\raggedright}m{\dimexpr 0.75in+0\tabcolsep}}{\textcolor[HTML]{000000}{\fontsize{11}{11}\selectfont{\global\setmainfont{Helvetica}{0}}}} & \multicolumn{1}{>{\raggedright}m{\dimexpr 0.75in+0\tabcolsep}}{\textcolor[HTML]{000000}{\fontsize{11}{11}\selectfont{\global\setmainfont{Helvetica}{0}}}} \\





\multicolumn{1}{>{\raggedleft}m{\dimexpr 0.75in+0\tabcolsep}}{\textcolor[HTML]{000000}{\fontsize{11}{11}\selectfont{\global\setmainfont{Helvetica}{1,967}}}} & \multicolumn{1}{>{\raggedright}m{\dimexpr 0.75in+0\tabcolsep}}{\textcolor[HTML]{000000}{\fontsize{11}{11}\selectfont{\global\setmainfont{Helvetica}{3902}}}} & \multicolumn{1}{>{\raggedright}m{\dimexpr 0.75in+0\tabcolsep}}{\textcolor[HTML]{000000}{\fontsize{11}{11}\selectfont{\global\setmainfont{Helvetica}{0}}}} & \multicolumn{1}{>{\raggedright}m{\dimexpr 0.75in+0\tabcolsep}}{\textcolor[HTML]{000000}{\fontsize{11}{11}\selectfont{\global\setmainfont{Helvetica}{0}}}} & \multicolumn{1}{>{\raggedright}m{\dimexpr 0.75in+0\tabcolsep}}{\textcolor[HTML]{000000}{\fontsize{11}{11}\selectfont{\global\setmainfont{Helvetica}{0}}}} \\





\multicolumn{1}{>{\raggedleft}m{\dimexpr 0.75in+0\tabcolsep}}{\textcolor[HTML]{000000}{\fontsize{11}{11}\selectfont{\global\setmainfont{Helvetica}{1,968}}}} & \multicolumn{1}{>{\raggedright}m{\dimexpr 0.75in+0\tabcolsep}}{\textcolor[HTML]{000000}{\fontsize{11}{11}\selectfont{\global\setmainfont{Helvetica}{1956}}}} & \multicolumn{1}{>{\raggedright}m{\dimexpr 0.75in+0\tabcolsep}}{\textcolor[HTML]{000000}{\fontsize{11}{11}\selectfont{\global\setmainfont{Helvetica}{0}}}} & \multicolumn{1}{>{\raggedright}m{\dimexpr 0.75in+0\tabcolsep}}{\textcolor[HTML]{000000}{\fontsize{11}{11}\selectfont{\global\setmainfont{Helvetica}{0}}}} & \multicolumn{1}{>{\raggedright}m{\dimexpr 0.75in+0\tabcolsep}}{\textcolor[HTML]{000000}{\fontsize{11}{11}\selectfont{\global\setmainfont{Helvetica}{0}}}} \\





\multicolumn{1}{>{\raggedleft}m{\dimexpr 0.75in+0\tabcolsep}}{\textcolor[HTML]{000000}{\fontsize{11}{11}\selectfont{\global\setmainfont{Helvetica}{1,969}}}} & \multicolumn{1}{>{\raggedright}m{\dimexpr 0.75in+0\tabcolsep}}{\textcolor[HTML]{000000}{\fontsize{11}{11}\selectfont{\global\setmainfont{Helvetica}{358}}}} & \multicolumn{1}{>{\raggedright}m{\dimexpr 0.75in+0\tabcolsep}}{\textcolor[HTML]{000000}{\fontsize{11}{11}\selectfont{\global\setmainfont{Helvetica}{0}}}} & \multicolumn{1}{>{\raggedright}m{\dimexpr 0.75in+0\tabcolsep}}{\textcolor[HTML]{000000}{\fontsize{11}{11}\selectfont{\global\setmainfont{Helvetica}{0}}}} & \multicolumn{1}{>{\raggedright}m{\dimexpr 0.75in+0\tabcolsep}}{\textcolor[HTML]{000000}{\fontsize{11}{11}\selectfont{\global\setmainfont{Helvetica}{0}}}} \\





\multicolumn{1}{>{\raggedleft}m{\dimexpr 0.75in+0\tabcolsep}}{\textcolor[HTML]{000000}{\fontsize{11}{11}\selectfont{\global\setmainfont{Helvetica}{1,970}}}} & \multicolumn{1}{>{\raggedright}m{\dimexpr 0.75in+0\tabcolsep}}{\textcolor[HTML]{000000}{\fontsize{11}{11}\selectfont{\global\setmainfont{Helvetica}{554}}}} & \multicolumn{1}{>{\raggedright}m{\dimexpr 0.75in+0\tabcolsep}}{\textcolor[HTML]{000000}{\fontsize{11}{11}\selectfont{\global\setmainfont{Helvetica}{0}}}} & \multicolumn{1}{>{\raggedright}m{\dimexpr 0.75in+0\tabcolsep}}{\textcolor[HTML]{000000}{\fontsize{11}{11}\selectfont{\global\setmainfont{Helvetica}{0}}}} & \multicolumn{1}{>{\raggedright}m{\dimexpr 0.75in+0\tabcolsep}}{\textcolor[HTML]{000000}{\fontsize{11}{11}\selectfont{\global\setmainfont{Helvetica}{0}}}} \\





\multicolumn{1}{>{\raggedleft}m{\dimexpr 0.75in+0\tabcolsep}}{\textcolor[HTML]{000000}{\fontsize{11}{11}\selectfont{\global\setmainfont{Helvetica}{1,971}}}} & \multicolumn{1}{>{\raggedright}m{\dimexpr 0.75in+0\tabcolsep}}{\textcolor[HTML]{000000}{\fontsize{11}{11}\selectfont{\global\setmainfont{Helvetica}{701}}}} & \multicolumn{1}{>{\raggedright}m{\dimexpr 0.75in+0\tabcolsep}}{\textcolor[HTML]{000000}{\fontsize{11}{11}\selectfont{\global\setmainfont{Helvetica}{0}}}} & \multicolumn{1}{>{\raggedright}m{\dimexpr 0.75in+0\tabcolsep}}{\textcolor[HTML]{000000}{\fontsize{11}{11}\selectfont{\global\setmainfont{Helvetica}{0}}}} & \multicolumn{1}{>{\raggedright}m{\dimexpr 0.75in+0\tabcolsep}}{\textcolor[HTML]{000000}{\fontsize{11}{11}\selectfont{\global\setmainfont{Helvetica}{0}}}} \\





\multicolumn{1}{>{\raggedleft}m{\dimexpr 0.75in+0\tabcolsep}}{\textcolor[HTML]{000000}{\fontsize{11}{11}\selectfont{\global\setmainfont{Helvetica}{1,972}}}} & \multicolumn{1}{>{\raggedright}m{\dimexpr 0.75in+0\tabcolsep}}{\textcolor[HTML]{000000}{\fontsize{11}{11}\selectfont{\global\setmainfont{Helvetica}{421}}}} & \multicolumn{1}{>{\raggedright}m{\dimexpr 0.75in+0\tabcolsep}}{\textcolor[HTML]{000000}{\fontsize{11}{11}\selectfont{\global\setmainfont{Helvetica}{0}}}} & \multicolumn{1}{>{\raggedright}m{\dimexpr 0.75in+0\tabcolsep}}{\textcolor[HTML]{000000}{\fontsize{11}{11}\selectfont{\global\setmainfont{Helvetica}{0}}}} & \multicolumn{1}{>{\raggedright}m{\dimexpr 0.75in+0\tabcolsep}}{\textcolor[HTML]{000000}{\fontsize{11}{11}\selectfont{\global\setmainfont{Helvetica}{0}}}} \\





\multicolumn{1}{>{\raggedleft}m{\dimexpr 0.75in+0\tabcolsep}}{\textcolor[HTML]{000000}{\fontsize{11}{11}\selectfont{\global\setmainfont{Helvetica}{1,973}}}} & \multicolumn{1}{>{\raggedright}m{\dimexpr 0.75in+0\tabcolsep}}{\textcolor[HTML]{000000}{\fontsize{11}{11}\selectfont{\global\setmainfont{Helvetica}{656}}}} & \multicolumn{1}{>{\raggedright}m{\dimexpr 0.75in+0\tabcolsep}}{\textcolor[HTML]{000000}{\fontsize{11}{11}\selectfont{\global\setmainfont{Helvetica}{0}}}} & \multicolumn{1}{>{\raggedright}m{\dimexpr 0.75in+0\tabcolsep}}{\textcolor[HTML]{000000}{\fontsize{11}{11}\selectfont{\global\setmainfont{Helvetica}{0}}}} & \multicolumn{1}{>{\raggedright}m{\dimexpr 0.75in+0\tabcolsep}}{\textcolor[HTML]{000000}{\fontsize{11}{11}\selectfont{\global\setmainfont{Helvetica}{0}}}} \\





\multicolumn{1}{>{\raggedleft}m{\dimexpr 0.75in+0\tabcolsep}}{\textcolor[HTML]{000000}{\fontsize{11}{11}\selectfont{\global\setmainfont{Helvetica}{1,974}}}} & \multicolumn{1}{>{\raggedright}m{\dimexpr 0.75in+0\tabcolsep}}{\textcolor[HTML]{000000}{\fontsize{11}{11}\selectfont{\global\setmainfont{Helvetica}{418}}}} & \multicolumn{1}{>{\raggedright}m{\dimexpr 0.75in+0\tabcolsep}}{\textcolor[HTML]{000000}{\fontsize{11}{11}\selectfont{\global\setmainfont{Helvetica}{0}}}} & \multicolumn{1}{>{\raggedright}m{\dimexpr 0.75in+0\tabcolsep}}{\textcolor[HTML]{000000}{\fontsize{11}{11}\selectfont{\global\setmainfont{Helvetica}{0}}}} & \multicolumn{1}{>{\raggedright}m{\dimexpr 0.75in+0\tabcolsep}}{\textcolor[HTML]{000000}{\fontsize{11}{11}\selectfont{\global\setmainfont{Helvetica}{0}}}} \\





\multicolumn{1}{>{\raggedleft}m{\dimexpr 0.75in+0\tabcolsep}}{\textcolor[HTML]{000000}{\fontsize{11}{11}\selectfont{\global\setmainfont{Helvetica}{1,975}}}} & \multicolumn{1}{>{\raggedright}m{\dimexpr 0.75in+0\tabcolsep}}{\textcolor[HTML]{000000}{\fontsize{11}{11}\selectfont{\global\setmainfont{Helvetica}{391.2}}}} & \multicolumn{1}{>{\raggedright}m{\dimexpr 0.75in+0\tabcolsep}}{\textcolor[HTML]{000000}{\fontsize{11}{11}\selectfont{\global\setmainfont{Helvetica}{0}}}} & \multicolumn{1}{>{\raggedright}m{\dimexpr 0.75in+0\tabcolsep}}{\textcolor[HTML]{000000}{\fontsize{11}{11}\selectfont{\global\setmainfont{Helvetica}{0}}}} & \multicolumn{1}{>{\raggedright}m{\dimexpr 0.75in+0\tabcolsep}}{\textcolor[HTML]{000000}{\fontsize{11}{11}\selectfont{\global\setmainfont{Helvetica}{0}}}} \\





\multicolumn{1}{>{\raggedleft}m{\dimexpr 0.75in+0\tabcolsep}}{\textcolor[HTML]{000000}{\fontsize{11}{11}\selectfont{\global\setmainfont{Helvetica}{1,976}}}} & \multicolumn{1}{>{\raggedright}m{\dimexpr 0.75in+0\tabcolsep}}{\textcolor[HTML]{000000}{\fontsize{11}{11}\selectfont{\global\setmainfont{Helvetica}{718.5}}}} & \multicolumn{1}{>{\raggedright}m{\dimexpr 0.75in+0\tabcolsep}}{\textcolor[HTML]{000000}{\fontsize{11}{11}\selectfont{\global\setmainfont{Helvetica}{0}}}} & \multicolumn{1}{>{\raggedright}m{\dimexpr 0.75in+0\tabcolsep}}{\textcolor[HTML]{000000}{\fontsize{11}{11}\selectfont{\global\setmainfont{Helvetica}{0}}}} & \multicolumn{1}{>{\raggedright}m{\dimexpr 0.75in+0\tabcolsep}}{\textcolor[HTML]{000000}{\fontsize{11}{11}\selectfont{\global\setmainfont{Helvetica}{0}}}} \\





\multicolumn{1}{>{\raggedleft}m{\dimexpr 0.75in+0\tabcolsep}}{\textcolor[HTML]{000000}{\fontsize{11}{11}\selectfont{\global\setmainfont{Helvetica}{1,977}}}} & \multicolumn{1}{>{\raggedright}m{\dimexpr 0.75in+0\tabcolsep}}{\textcolor[HTML]{000000}{\fontsize{11}{11}\selectfont{\global\setmainfont{Helvetica}{119.3}}}} & \multicolumn{1}{>{\raggedright}m{\dimexpr 0.75in+0\tabcolsep}}{\textcolor[HTML]{000000}{\fontsize{11}{11}\selectfont{\global\setmainfont{Helvetica}{0}}}} & \multicolumn{1}{>{\raggedright}m{\dimexpr 0.75in+0\tabcolsep}}{\textcolor[HTML]{000000}{\fontsize{11}{11}\selectfont{\global\setmainfont{Helvetica}{0}}}} & \multicolumn{1}{>{\raggedright}m{\dimexpr 0.75in+0\tabcolsep}}{\textcolor[HTML]{000000}{\fontsize{11}{11}\selectfont{\global\setmainfont{Helvetica}{0}}}} \\





\multicolumn{1}{>{\raggedleft}m{\dimexpr 0.75in+0\tabcolsep}}{\textcolor[HTML]{000000}{\fontsize{11}{11}\selectfont{\global\setmainfont{Helvetica}{1,978}}}} & \multicolumn{1}{>{\raggedright}m{\dimexpr 0.75in+0\tabcolsep}}{\textcolor[HTML]{000000}{\fontsize{11}{11}\selectfont{\global\setmainfont{Helvetica}{191.9}}}} & \multicolumn{1}{>{\raggedright}m{\dimexpr 0.75in+0\tabcolsep}}{\textcolor[HTML]{000000}{\fontsize{11}{11}\selectfont{\global\setmainfont{Helvetica}{0}}}} & \multicolumn{1}{>{\raggedright}m{\dimexpr 0.75in+0\tabcolsep}}{\textcolor[HTML]{000000}{\fontsize{11}{11}\selectfont{\global\setmainfont{Helvetica}{0}}}} & \multicolumn{1}{>{\raggedright}m{\dimexpr 0.75in+0\tabcolsep}}{\textcolor[HTML]{000000}{\fontsize{11}{11}\selectfont{\global\setmainfont{Helvetica}{0}}}} \\





\multicolumn{1}{>{\raggedleft}m{\dimexpr 0.75in+0\tabcolsep}}{\textcolor[HTML]{000000}{\fontsize{11}{11}\selectfont{\global\setmainfont{Helvetica}{1,979}}}} & \multicolumn{1}{>{\raggedright}m{\dimexpr 0.75in+0\tabcolsep}}{\textcolor[HTML]{000000}{\fontsize{11}{11}\selectfont{\global\setmainfont{Helvetica}{197.9}}}} & \multicolumn{1}{>{\raggedright}m{\dimexpr 0.75in+0\tabcolsep}}{\textcolor[HTML]{000000}{\fontsize{11}{11}\selectfont{\global\setmainfont{Helvetica}{0}}}} & \multicolumn{1}{>{\raggedright}m{\dimexpr 0.75in+0\tabcolsep}}{\textcolor[HTML]{000000}{\fontsize{11}{11}\selectfont{\global\setmainfont{Helvetica}{0}}}} & \multicolumn{1}{>{\raggedright}m{\dimexpr 0.75in+0\tabcolsep}}{\textcolor[HTML]{000000}{\fontsize{11}{11}\selectfont{\global\setmainfont{Helvetica}{0}}}} \\





\multicolumn{1}{>{\raggedleft}m{\dimexpr 0.75in+0\tabcolsep}}{\textcolor[HTML]{000000}{\fontsize{11}{11}\selectfont{\global\setmainfont{Helvetica}{1,980}}}} & \multicolumn{1}{>{\raggedright}m{\dimexpr 0.75in+0\tabcolsep}}{\textcolor[HTML]{000000}{\fontsize{11}{11}\selectfont{\global\setmainfont{Helvetica}{272}}}} & \multicolumn{1}{>{\raggedright}m{\dimexpr 0.75in+0\tabcolsep}}{\textcolor[HTML]{000000}{\fontsize{11}{11}\selectfont{\global\setmainfont{Helvetica}{0}}}} & \multicolumn{1}{>{\raggedright}m{\dimexpr 0.75in+0\tabcolsep}}{\textcolor[HTML]{000000}{\fontsize{11}{11}\selectfont{\global\setmainfont{Helvetica}{0}}}} & \multicolumn{1}{>{\raggedright}m{\dimexpr 0.75in+0\tabcolsep}}{\textcolor[HTML]{000000}{\fontsize{11}{11}\selectfont{\global\setmainfont{Helvetica}{0}}}} \\





\multicolumn{1}{>{\raggedleft}m{\dimexpr 0.75in+0\tabcolsep}}{\textcolor[HTML]{000000}{\fontsize{11}{11}\selectfont{\global\setmainfont{Helvetica}{1,981}}}} & \multicolumn{1}{>{\raggedright}m{\dimexpr 0.75in+0\tabcolsep}}{\textcolor[HTML]{000000}{\fontsize{11}{11}\selectfont{\global\setmainfont{Helvetica}{227.9}}}} & \multicolumn{1}{>{\raggedright}m{\dimexpr 0.75in+0\tabcolsep}}{\textcolor[HTML]{000000}{\fontsize{11}{11}\selectfont{\global\setmainfont{Helvetica}{0}}}} & \multicolumn{1}{>{\raggedright}m{\dimexpr 0.75in+0\tabcolsep}}{\textcolor[HTML]{000000}{\fontsize{11}{11}\selectfont{\global\setmainfont{Helvetica}{0}}}} & \multicolumn{1}{>{\raggedright}m{\dimexpr 0.75in+0\tabcolsep}}{\textcolor[HTML]{000000}{\fontsize{11}{11}\selectfont{\global\setmainfont{Helvetica}{0}}}} \\





\multicolumn{1}{>{\raggedleft}m{\dimexpr 0.75in+0\tabcolsep}}{\textcolor[HTML]{000000}{\fontsize{11}{11}\selectfont{\global\setmainfont{Helvetica}{1,982}}}} & \multicolumn{1}{>{\raggedright}m{\dimexpr 0.75in+0\tabcolsep}}{\textcolor[HTML]{000000}{\fontsize{11}{11}\selectfont{\global\setmainfont{Helvetica}{157.5}}}} & \multicolumn{1}{>{\raggedright}m{\dimexpr 0.75in+0\tabcolsep}}{\textcolor[HTML]{000000}{\fontsize{11}{11}\selectfont{\global\setmainfont{Helvetica}{0}}}} & \multicolumn{1}{>{\raggedright}m{\dimexpr 0.75in+0\tabcolsep}}{\textcolor[HTML]{000000}{\fontsize{11}{11}\selectfont{\global\setmainfont{Helvetica}{0}}}} & \multicolumn{1}{>{\raggedright}m{\dimexpr 0.75in+0\tabcolsep}}{\textcolor[HTML]{000000}{\fontsize{11}{11}\selectfont{\global\setmainfont{Helvetica}{0}}}} \\





\multicolumn{1}{>{\raggedleft}m{\dimexpr 0.75in+0\tabcolsep}}{\textcolor[HTML]{000000}{\fontsize{11}{11}\selectfont{\global\setmainfont{Helvetica}{1,983}}}} & \multicolumn{1}{>{\raggedright}m{\dimexpr 0.75in+0\tabcolsep}}{\textcolor[HTML]{000000}{\fontsize{11}{11}\selectfont{\global\setmainfont{Helvetica}{131.5}}}} & \multicolumn{1}{>{\raggedright}m{\dimexpr 0.75in+0\tabcolsep}}{\textcolor[HTML]{000000}{\fontsize{11}{11}\selectfont{\global\setmainfont{Helvetica}{0}}}} & \multicolumn{1}{>{\raggedright}m{\dimexpr 0.75in+0\tabcolsep}}{\textcolor[HTML]{000000}{\fontsize{11}{11}\selectfont{\global\setmainfont{Helvetica}{0}}}} & \multicolumn{1}{>{\raggedright}m{\dimexpr 0.75in+0\tabcolsep}}{\textcolor[HTML]{000000}{\fontsize{11}{11}\selectfont{\global\setmainfont{Helvetica}{0}}}} \\





\multicolumn{1}{>{\raggedleft}m{\dimexpr 0.75in+0\tabcolsep}}{\textcolor[HTML]{000000}{\fontsize{11}{11}\selectfont{\global\setmainfont{Helvetica}{1,984}}}} & \multicolumn{1}{>{\raggedright}m{\dimexpr 0.75in+0\tabcolsep}}{\textcolor[HTML]{000000}{\fontsize{11}{11}\selectfont{\global\setmainfont{Helvetica}{294.7}}}} & \multicolumn{1}{>{\raggedright}m{\dimexpr 0.75in+0\tabcolsep}}{\textcolor[HTML]{000000}{\fontsize{11}{11}\selectfont{\global\setmainfont{Helvetica}{0}}}} & \multicolumn{1}{>{\raggedright}m{\dimexpr 0.75in+0\tabcolsep}}{\textcolor[HTML]{000000}{\fontsize{11}{11}\selectfont{\global\setmainfont{Helvetica}{0}}}} & \multicolumn{1}{>{\raggedright}m{\dimexpr 0.75in+0\tabcolsep}}{\textcolor[HTML]{000000}{\fontsize{11}{11}\selectfont{\global\setmainfont{Helvetica}{0}}}} \\





\multicolumn{1}{>{\raggedleft}m{\dimexpr 0.75in+0\tabcolsep}}{\textcolor[HTML]{000000}{\fontsize{11}{11}\selectfont{\global\setmainfont{Helvetica}{1,985}}}} & \multicolumn{1}{>{\raggedright}m{\dimexpr 0.75in+0\tabcolsep}}{\textcolor[HTML]{000000}{\fontsize{11}{11}\selectfont{\global\setmainfont{Helvetica}{182.6}}}} & \multicolumn{1}{>{\raggedright}m{\dimexpr 0.75in+0\tabcolsep}}{\textcolor[HTML]{000000}{\fontsize{11}{11}\selectfont{\global\setmainfont{Helvetica}{0}}}} & \multicolumn{1}{>{\raggedright}m{\dimexpr 0.75in+0\tabcolsep}}{\textcolor[HTML]{000000}{\fontsize{11}{11}\selectfont{\global\setmainfont{Helvetica}{0}}}} & \multicolumn{1}{>{\raggedright}m{\dimexpr 0.75in+0\tabcolsep}}{\textcolor[HTML]{000000}{\fontsize{11}{11}\selectfont{\global\setmainfont{Helvetica}{0}}}} \\





\multicolumn{1}{>{\raggedleft}m{\dimexpr 0.75in+0\tabcolsep}}{\textcolor[HTML]{000000}{\fontsize{11}{11}\selectfont{\global\setmainfont{Helvetica}{1,986}}}} & \multicolumn{1}{>{\raggedright}m{\dimexpr 0.75in+0\tabcolsep}}{\textcolor[HTML]{000000}{\fontsize{11}{11}\selectfont{\global\setmainfont{Helvetica}{256.8}}}} & \multicolumn{1}{>{\raggedright}m{\dimexpr 0.75in+0\tabcolsep}}{\textcolor[HTML]{000000}{\fontsize{11}{11}\selectfont{\global\setmainfont{Helvetica}{0}}}} & \multicolumn{1}{>{\raggedright}m{\dimexpr 0.75in+0\tabcolsep}}{\textcolor[HTML]{000000}{\fontsize{11}{11}\selectfont{\global\setmainfont{Helvetica}{0}}}} & \multicolumn{1}{>{\raggedright}m{\dimexpr 0.75in+0\tabcolsep}}{\textcolor[HTML]{000000}{\fontsize{11}{11}\selectfont{\global\setmainfont{Helvetica}{0}}}} \\





\multicolumn{1}{>{\raggedleft}m{\dimexpr 0.75in+0\tabcolsep}}{\textcolor[HTML]{000000}{\fontsize{11}{11}\selectfont{\global\setmainfont{Helvetica}{1,987}}}} & \multicolumn{1}{>{\raggedright}m{\dimexpr 0.75in+0\tabcolsep}}{\textcolor[HTML]{000000}{\fontsize{11}{11}\selectfont{\global\setmainfont{Helvetica}{181.3}}}} & \multicolumn{1}{>{\raggedright}m{\dimexpr 0.75in+0\tabcolsep}}{\textcolor[HTML]{000000}{\fontsize{11}{11}\selectfont{\global\setmainfont{Helvetica}{0}}}} & \multicolumn{1}{>{\raggedright}m{\dimexpr 0.75in+0\tabcolsep}}{\textcolor[HTML]{000000}{\fontsize{11}{11}\selectfont{\global\setmainfont{Helvetica}{0}}}} & \multicolumn{1}{>{\raggedright}m{\dimexpr 0.75in+0\tabcolsep}}{\textcolor[HTML]{000000}{\fontsize{11}{11}\selectfont{\global\setmainfont{Helvetica}{0}}}} \\





\multicolumn{1}{>{\raggedleft}m{\dimexpr 0.75in+0\tabcolsep}}{\textcolor[HTML]{000000}{\fontsize{11}{11}\selectfont{\global\setmainfont{Helvetica}{1,988}}}} & \multicolumn{1}{>{\raggedright}m{\dimexpr 0.75in+0\tabcolsep}}{\textcolor[HTML]{000000}{\fontsize{11}{11}\selectfont{\global\setmainfont{Helvetica}{231.6}}}} & \multicolumn{1}{>{\raggedright}m{\dimexpr 0.75in+0\tabcolsep}}{\textcolor[HTML]{000000}{\fontsize{11}{11}\selectfont{\global\setmainfont{Helvetica}{0}}}} & \multicolumn{1}{>{\raggedright}m{\dimexpr 0.75in+0\tabcolsep}}{\textcolor[HTML]{000000}{\fontsize{11}{11}\selectfont{\global\setmainfont{Helvetica}{0}}}} & \multicolumn{1}{>{\raggedright}m{\dimexpr 0.75in+0\tabcolsep}}{\textcolor[HTML]{000000}{\fontsize{11}{11}\selectfont{\global\setmainfont{Helvetica}{0}}}} \\





\multicolumn{1}{>{\raggedleft}m{\dimexpr 0.75in+0\tabcolsep}}{\textcolor[HTML]{000000}{\fontsize{11}{11}\selectfont{\global\setmainfont{Helvetica}{1,989}}}} & \multicolumn{1}{>{\raggedright}m{\dimexpr 0.75in+0\tabcolsep}}{\textcolor[HTML]{000000}{\fontsize{11}{11}\selectfont{\global\setmainfont{Helvetica}{212}}}} & \multicolumn{1}{>{\raggedright}m{\dimexpr 0.75in+0\tabcolsep}}{\textcolor[HTML]{000000}{\fontsize{11}{11}\selectfont{\global\setmainfont{Helvetica}{0}}}} & \multicolumn{1}{>{\raggedright}m{\dimexpr 0.75in+0\tabcolsep}}{\textcolor[HTML]{000000}{\fontsize{11}{11}\selectfont{\global\setmainfont{Helvetica}{0}}}} & \multicolumn{1}{>{\raggedright}m{\dimexpr 0.75in+0\tabcolsep}}{\textcolor[HTML]{000000}{\fontsize{11}{11}\selectfont{\global\setmainfont{Helvetica}{0}}}} \\





\multicolumn{1}{>{\raggedleft}m{\dimexpr 0.75in+0\tabcolsep}}{\textcolor[HTML]{000000}{\fontsize{11}{11}\selectfont{\global\setmainfont{Helvetica}{1,990}}}} & \multicolumn{1}{>{\raggedright}m{\dimexpr 0.75in+0\tabcolsep}}{\textcolor[HTML]{000000}{\fontsize{11}{11}\selectfont{\global\setmainfont{Helvetica}{230.2}}}} & \multicolumn{1}{>{\raggedright}m{\dimexpr 0.75in+0\tabcolsep}}{\textcolor[HTML]{000000}{\fontsize{11}{11}\selectfont{\global\setmainfont{Helvetica}{0}}}} & \multicolumn{1}{>{\raggedright}m{\dimexpr 0.75in+0\tabcolsep}}{\textcolor[HTML]{000000}{\fontsize{11}{11}\selectfont{\global\setmainfont{Helvetica}{0}}}} & \multicolumn{1}{>{\raggedright}m{\dimexpr 0.75in+0\tabcolsep}}{\textcolor[HTML]{000000}{\fontsize{11}{11}\selectfont{\global\setmainfont{Helvetica}{0}}}} \\





\multicolumn{1}{>{\raggedleft}m{\dimexpr 0.75in+0\tabcolsep}}{\textcolor[HTML]{000000}{\fontsize{11}{11}\selectfont{\global\setmainfont{Helvetica}{1,991}}}} & \multicolumn{1}{>{\raggedright}m{\dimexpr 0.75in+0\tabcolsep}}{\textcolor[HTML]{000000}{\fontsize{11}{11}\selectfont{\global\setmainfont{Helvetica}{471.3}}}} & \multicolumn{1}{>{\raggedright}m{\dimexpr 0.75in+0\tabcolsep}}{\textcolor[HTML]{000000}{\fontsize{11}{11}\selectfont{\global\setmainfont{Helvetica}{42.7}}}} & \multicolumn{1}{>{\raggedright}m{\dimexpr 0.75in+0\tabcolsep}}{\textcolor[HTML]{000000}{\fontsize{11}{11}\selectfont{\global\setmainfont{Helvetica}{39}}}} & \multicolumn{1}{>{\raggedright}m{\dimexpr 0.75in+0\tabcolsep}}{\textcolor[HTML]{000000}{\fontsize{11}{11}\selectfont{\global\setmainfont{Helvetica}{9.3}}}} \\





\multicolumn{1}{>{\raggedleft}m{\dimexpr 0.75in+0\tabcolsep}}{\textcolor[HTML]{000000}{\fontsize{11}{11}\selectfont{\global\setmainfont{Helvetica}{1,992}}}} & \multicolumn{1}{>{\raggedright}m{\dimexpr 0.75in+0\tabcolsep}}{\textcolor[HTML]{000000}{\fontsize{11}{11}\selectfont{\global\setmainfont{Helvetica}{389.6}}}} & \multicolumn{1}{>{\raggedright}m{\dimexpr 0.75in+0\tabcolsep}}{\textcolor[HTML]{000000}{\fontsize{11}{11}\selectfont{\global\setmainfont{Helvetica}{13.5}}}} & \multicolumn{1}{>{\raggedright}m{\dimexpr 0.75in+0\tabcolsep}}{\textcolor[HTML]{000000}{\fontsize{11}{11}\selectfont{\global\setmainfont{Helvetica}{42.1}}}} & \multicolumn{1}{>{\raggedright}m{\dimexpr 0.75in+0\tabcolsep}}{\textcolor[HTML]{000000}{\fontsize{11}{11}\selectfont{\global\setmainfont{Helvetica}{6.2}}}} \\





\multicolumn{1}{>{\raggedleft}m{\dimexpr 0.75in+0\tabcolsep}}{\textcolor[HTML]{000000}{\fontsize{11}{11}\selectfont{\global\setmainfont{Helvetica}{1,993}}}} & \multicolumn{1}{>{\raggedright}m{\dimexpr 0.75in+0\tabcolsep}}{\textcolor[HTML]{000000}{\fontsize{11}{11}\selectfont{\global\setmainfont{Helvetica}{173.2}}}} & \multicolumn{1}{>{\raggedright}m{\dimexpr 0.75in+0\tabcolsep}}{\textcolor[HTML]{000000}{\fontsize{11}{11}\selectfont{\global\setmainfont{Helvetica}{0.4}}}} & \multicolumn{1}{>{\raggedright}m{\dimexpr 0.75in+0\tabcolsep}}{\textcolor[HTML]{000000}{\fontsize{11}{11}\selectfont{\global\setmainfont{Helvetica}{91.2}}}} & \multicolumn{1}{>{\raggedright}m{\dimexpr 0.75in+0\tabcolsep}}{\textcolor[HTML]{000000}{\fontsize{11}{11}\selectfont{\global\setmainfont{Helvetica}{11}}}} \\





\multicolumn{1}{>{\raggedleft}m{\dimexpr 0.75in+0\tabcolsep}}{\textcolor[HTML]{000000}{\fontsize{11}{11}\selectfont{\global\setmainfont{Helvetica}{1,994}}}} & \multicolumn{1}{>{\raggedright}m{\dimexpr 0.75in+0\tabcolsep}}{\textcolor[HTML]{000000}{\fontsize{11}{11}\selectfont{\global\setmainfont{Helvetica}{370.7}}}} & \multicolumn{1}{>{\raggedright}m{\dimexpr 0.75in+0\tabcolsep}}{\textcolor[HTML]{000000}{\fontsize{11}{11}\selectfont{\global\setmainfont{Helvetica}{2.1}}}} & \multicolumn{1}{>{\raggedright}m{\dimexpr 0.75in+0\tabcolsep}}{\textcolor[HTML]{000000}{\fontsize{11}{11}\selectfont{\global\setmainfont{Helvetica}{210.8}}}} & \multicolumn{1}{>{\raggedright}m{\dimexpr 0.75in+0\tabcolsep}}{\textcolor[HTML]{000000}{\fontsize{11}{11}\selectfont{\global\setmainfont{Helvetica}{28.6}}}} \\





\multicolumn{1}{>{\raggedleft}m{\dimexpr 0.75in+0\tabcolsep}}{\textcolor[HTML]{000000}{\fontsize{11}{11}\selectfont{\global\setmainfont{Helvetica}{1,995}}}} & \multicolumn{1}{>{\raggedright}m{\dimexpr 0.75in+0\tabcolsep}}{\textcolor[HTML]{000000}{\fontsize{11}{11}\selectfont{\global\setmainfont{Helvetica}{228.6}}}} & \multicolumn{1}{>{\raggedright}m{\dimexpr 0.75in+0\tabcolsep}}{\textcolor[HTML]{000000}{\fontsize{11}{11}\selectfont{\global\setmainfont{Helvetica}{7.2}}}} & \multicolumn{1}{>{\raggedright}m{\dimexpr 0.75in+0\tabcolsep}}{\textcolor[HTML]{000000}{\fontsize{11}{11}\selectfont{\global\setmainfont{Helvetica}{192.1}}}} & \multicolumn{1}{>{\raggedright}m{\dimexpr 0.75in+0\tabcolsep}}{\textcolor[HTML]{000000}{\fontsize{11}{11}\selectfont{\global\setmainfont{Helvetica}{36.8}}}} \\





\multicolumn{1}{>{\raggedleft}m{\dimexpr 0.75in+0\tabcolsep}}{\textcolor[HTML]{000000}{\fontsize{11}{11}\selectfont{\global\setmainfont{Helvetica}{1,996}}}} & \multicolumn{1}{>{\raggedright}m{\dimexpr 0.75in+0\tabcolsep}}{\textcolor[HTML]{000000}{\fontsize{11}{11}\selectfont{\global\setmainfont{Helvetica}{252.2}}}} & \multicolumn{1}{>{\raggedright}m{\dimexpr 0.75in+0\tabcolsep}}{\textcolor[HTML]{000000}{\fontsize{11}{11}\selectfont{\global\setmainfont{Helvetica}{5.7}}}} & \multicolumn{1}{>{\raggedright}m{\dimexpr 0.75in+0\tabcolsep}}{\textcolor[HTML]{000000}{\fontsize{11}{11}\selectfont{\global\setmainfont{Helvetica}{475.1}}}} & \multicolumn{1}{>{\raggedright}m{\dimexpr 0.75in+0\tabcolsep}}{\textcolor[HTML]{000000}{\fontsize{11}{11}\selectfont{\global\setmainfont{Helvetica}{104.7}}}} \\





\multicolumn{1}{>{\raggedleft}m{\dimexpr 0.75in+0\tabcolsep}}{\textcolor[HTML]{000000}{\fontsize{11}{11}\selectfont{\global\setmainfont{Helvetica}{1,997}}}} & \multicolumn{1}{>{\raggedright}m{\dimexpr 0.75in+0\tabcolsep}}{\textcolor[HTML]{000000}{\fontsize{11}{11}\selectfont{\global\setmainfont{Helvetica}{215.5}}}} & \multicolumn{1}{>{\raggedright}m{\dimexpr 0.75in+0\tabcolsep}}{\textcolor[HTML]{000000}{\fontsize{11}{11}\selectfont{\global\setmainfont{Helvetica}{7.2}}}} & \multicolumn{1}{>{\raggedright}m{\dimexpr 0.75in+0\tabcolsep}}{\textcolor[HTML]{000000}{\fontsize{11}{11}\selectfont{\global\setmainfont{Helvetica}{133.9}}}} & \multicolumn{1}{>{\raggedright}m{\dimexpr 0.75in+0\tabcolsep}}{\textcolor[HTML]{000000}{\fontsize{11}{11}\selectfont{\global\setmainfont{Helvetica}{22.1}}}} \\





\multicolumn{1}{>{\raggedleft}m{\dimexpr 0.75in+0\tabcolsep}}{\textcolor[HTML]{000000}{\fontsize{11}{11}\selectfont{\global\setmainfont{Helvetica}{1,998}}}} & \multicolumn{1}{>{\raggedright}m{\dimexpr 0.75in+0\tabcolsep}}{\textcolor[HTML]{000000}{\fontsize{11}{11}\selectfont{\global\setmainfont{Helvetica}{268.5}}}} & \multicolumn{1}{>{\raggedright}m{\dimexpr 0.75in+0\tabcolsep}}{\textcolor[HTML]{000000}{\fontsize{11}{11}\selectfont{\global\setmainfont{Helvetica}{40.4}}}} & \multicolumn{1}{>{\raggedright}m{\dimexpr 0.75in+0\tabcolsep}}{\textcolor[HTML]{000000}{\fontsize{11}{11}\selectfont{\global\setmainfont{Helvetica}{278}}}} & \multicolumn{1}{>{\raggedright}m{\dimexpr 0.75in+0\tabcolsep}}{\textcolor[HTML]{000000}{\fontsize{11}{11}\selectfont{\global\setmainfont{Helvetica}{28.1}}}} \\





\multicolumn{1}{>{\raggedleft}m{\dimexpr 0.75in+0\tabcolsep}}{\textcolor[HTML]{000000}{\fontsize{11}{11}\selectfont{\global\setmainfont{Helvetica}{1,999}}}} & \multicolumn{1}{>{\raggedright}m{\dimexpr 0.75in+0\tabcolsep}}{\textcolor[HTML]{000000}{\fontsize{11}{11}\selectfont{\global\setmainfont{Helvetica}{191.8}}}} & \multicolumn{1}{>{\raggedright}m{\dimexpr 0.75in+0\tabcolsep}}{\textcolor[HTML]{000000}{\fontsize{11}{11}\selectfont{\global\setmainfont{Helvetica}{12.7}}}} & \multicolumn{1}{>{\raggedright}m{\dimexpr 0.75in+0\tabcolsep}}{\textcolor[HTML]{000000}{\fontsize{11}{11}\selectfont{\global\setmainfont{Helvetica}{166.4}}}} & \multicolumn{1}{>{\raggedright}m{\dimexpr 0.75in+0\tabcolsep}}{\textcolor[HTML]{000000}{\fontsize{11}{11}\selectfont{\global\setmainfont{Helvetica}{15.2}}}} \\





\multicolumn{1}{>{\raggedleft}m{\dimexpr 0.75in+0\tabcolsep}}{\textcolor[HTML]{000000}{\fontsize{11}{11}\selectfont{\global\setmainfont{Helvetica}{2,000}}}} & \multicolumn{1}{>{\raggedright}m{\dimexpr 0.75in+0\tabcolsep}}{\textcolor[HTML]{000000}{\fontsize{11}{11}\selectfont{\global\setmainfont{Helvetica}{205.4}}}} & \multicolumn{1}{>{\raggedright}m{\dimexpr 0.75in+0\tabcolsep}}{\textcolor[HTML]{000000}{\fontsize{11}{11}\selectfont{\global\setmainfont{Helvetica}{7.7}}}} & \multicolumn{1}{>{\raggedright}m{\dimexpr 0.75in+0\tabcolsep}}{\textcolor[HTML]{000000}{\fontsize{11}{11}\selectfont{\global\setmainfont{Helvetica}{70.9}}}} & \multicolumn{1}{>{\raggedright}m{\dimexpr 0.75in+0\tabcolsep}}{\textcolor[HTML]{000000}{\fontsize{11}{11}\selectfont{\global\setmainfont{Helvetica}{4.7}}}} \\





\multicolumn{1}{>{\raggedleft}m{\dimexpr 0.75in+0\tabcolsep}}{\textcolor[HTML]{000000}{\fontsize{11}{11}\selectfont{\global\setmainfont{Helvetica}{2,001}}}} & \multicolumn{1}{>{\raggedright}m{\dimexpr 0.75in+0\tabcolsep}}{\textcolor[HTML]{000000}{\fontsize{11}{11}\selectfont{\global\setmainfont{Helvetica}{174}}}} & \multicolumn{1}{>{\raggedright}m{\dimexpr 0.75in+0\tabcolsep}}{\textcolor[HTML]{000000}{\fontsize{11}{11}\selectfont{\global\setmainfont{Helvetica}{9.2}}}} & \multicolumn{1}{>{\raggedright}m{\dimexpr 0.75in+0\tabcolsep}}{\textcolor[HTML]{000000}{\fontsize{11}{11}\selectfont{\global\setmainfont{Helvetica}{26.4}}}} & \multicolumn{1}{>{\raggedright}m{\dimexpr 0.75in+0\tabcolsep}}{\textcolor[HTML]{000000}{\fontsize{11}{11}\selectfont{\global\setmainfont{Helvetica}{9}}}} \\





\multicolumn{1}{>{\raggedleft}m{\dimexpr 0.75in+0\tabcolsep}}{\textcolor[HTML]{000000}{\fontsize{11}{11}\selectfont{\global\setmainfont{Helvetica}{2,002}}}} & \multicolumn{1}{>{\raggedright}m{\dimexpr 0.75in+0\tabcolsep}}{\textcolor[HTML]{000000}{\fontsize{11}{11}\selectfont{\global\setmainfont{Helvetica}{154.9}}}} & \multicolumn{1}{>{\raggedright}m{\dimexpr 0.75in+0\tabcolsep}}{\textcolor[HTML]{000000}{\fontsize{11}{11}\selectfont{\global\setmainfont{Helvetica}{1.2}}}} & \multicolumn{1}{>{\raggedright}m{\dimexpr 0.75in+0\tabcolsep}}{\textcolor[HTML]{000000}{\fontsize{11}{11}\selectfont{\global\setmainfont{Helvetica}{2.6}}}} & \multicolumn{1}{>{\raggedright}m{\dimexpr 0.75in+0\tabcolsep}}{\textcolor[HTML]{000000}{\fontsize{11}{11}\selectfont{\global\setmainfont{Helvetica}{1.4}}}} \\





\multicolumn{1}{>{\raggedleft}m{\dimexpr 0.75in+0\tabcolsep}}{\textcolor[HTML]{000000}{\fontsize{11}{11}\selectfont{\global\setmainfont{Helvetica}{2,003}}}} & \multicolumn{1}{>{\raggedright}m{\dimexpr 0.75in+0\tabcolsep}}{\textcolor[HTML]{000000}{\fontsize{11}{11}\selectfont{\global\setmainfont{Helvetica}{14.5}}}} & \multicolumn{1}{>{\raggedright}m{\dimexpr 0.75in+0\tabcolsep}}{\textcolor[HTML]{000000}{\fontsize{11}{11}\selectfont{\global\setmainfont{Helvetica}{0.4}}}} & \multicolumn{1}{>{\raggedright}m{\dimexpr 0.75in+0\tabcolsep}}{\textcolor[HTML]{000000}{\fontsize{11}{11}\selectfont{\global\setmainfont{Helvetica}{7.6}}}} & \multicolumn{1}{>{\raggedright}m{\dimexpr 0.75in+0\tabcolsep}}{\textcolor[HTML]{000000}{\fontsize{11}{11}\selectfont{\global\setmainfont{Helvetica}{4.6}}}} \\





\multicolumn{1}{>{\raggedleft}m{\dimexpr 0.75in+0\tabcolsep}}{\textcolor[HTML]{000000}{\fontsize{11}{11}\selectfont{\global\setmainfont{Helvetica}{2,004}}}} & \multicolumn{1}{>{\raggedright}m{\dimexpr 0.75in+0\tabcolsep}}{\textcolor[HTML]{000000}{\fontsize{11}{11}\selectfont{\global\setmainfont{Helvetica}{21.2}}}} & \multicolumn{1}{>{\raggedright}m{\dimexpr 0.75in+0\tabcolsep}}{\textcolor[HTML]{000000}{\fontsize{11}{11}\selectfont{\global\setmainfont{Helvetica}{7.4}}}} & \multicolumn{1}{>{\raggedright}m{\dimexpr 0.75in+0\tabcolsep}}{\textcolor[HTML]{000000}{\fontsize{11}{11}\selectfont{\global\setmainfont{Helvetica}{12.4}}}} & \multicolumn{1}{>{\raggedright}m{\dimexpr 0.75in+0\tabcolsep}}{\textcolor[HTML]{000000}{\fontsize{11}{11}\selectfont{\global\setmainfont{Helvetica}{8.5}}}} \\





\multicolumn{1}{>{\raggedleft}m{\dimexpr 0.75in+0\tabcolsep}}{\textcolor[HTML]{000000}{\fontsize{11}{11}\selectfont{\global\setmainfont{Helvetica}{2,005}}}} & \multicolumn{1}{>{\raggedright}m{\dimexpr 0.75in+0\tabcolsep}}{\textcolor[HTML]{000000}{\fontsize{11}{11}\selectfont{\global\setmainfont{Helvetica}{80.1}}}} & \multicolumn{1}{>{\raggedright}m{\dimexpr 0.75in+0\tabcolsep}}{\textcolor[HTML]{000000}{\fontsize{11}{11}\selectfont{\global\setmainfont{Helvetica}{5.2}}}} & \multicolumn{1}{>{\raggedright}m{\dimexpr 0.75in+0\tabcolsep}}{\textcolor[HTML]{000000}{\fontsize{11}{11}\selectfont{\global\setmainfont{Helvetica}{59.1}}}} & \multicolumn{1}{>{\raggedright}m{\dimexpr 0.75in+0\tabcolsep}}{\textcolor[HTML]{000000}{\fontsize{11}{11}\selectfont{\global\setmainfont{Helvetica}{13.6}}}} \\





\multicolumn{1}{>{\raggedleft}m{\dimexpr 0.75in+0\tabcolsep}}{\textcolor[HTML]{000000}{\fontsize{11}{11}\selectfont{\global\setmainfont{Helvetica}{2,006}}}} & \multicolumn{1}{>{\raggedright}m{\dimexpr 0.75in+0\tabcolsep}}{\textcolor[HTML]{000000}{\fontsize{11}{11}\selectfont{\global\setmainfont{Helvetica}{143}}}} & \multicolumn{1}{>{\raggedright}m{\dimexpr 0.75in+0\tabcolsep}}{\textcolor[HTML]{000000}{\fontsize{11}{11}\selectfont{\global\setmainfont{Helvetica}{3.6}}}} & \multicolumn{1}{>{\raggedright}m{\dimexpr 0.75in+0\tabcolsep}}{\textcolor[HTML]{000000}{\fontsize{11}{11}\selectfont{\global\setmainfont{Helvetica}{11.3}}}} & \multicolumn{1}{>{\raggedright}m{\dimexpr 0.75in+0\tabcolsep}}{\textcolor[HTML]{000000}{\fontsize{11}{11}\selectfont{\global\setmainfont{Helvetica}{35.3}}}} \\





\multicolumn{1}{>{\raggedleft}m{\dimexpr 0.75in+0\tabcolsep}}{\textcolor[HTML]{000000}{\fontsize{11}{11}\selectfont{\global\setmainfont{Helvetica}{2,007}}}} & \multicolumn{1}{>{\raggedright}m{\dimexpr 0.75in+0\tabcolsep}}{\textcolor[HTML]{000000}{\fontsize{11}{11}\selectfont{\global\setmainfont{Helvetica}{146}}}} & \multicolumn{1}{>{\raggedright}m{\dimexpr 0.75in+0\tabcolsep}}{\textcolor[HTML]{000000}{\fontsize{11}{11}\selectfont{\global\setmainfont{Helvetica}{1}}}} & \multicolumn{1}{>{\raggedright}m{\dimexpr 0.75in+0\tabcolsep}}{\textcolor[HTML]{000000}{\fontsize{11}{11}\selectfont{\global\setmainfont{Helvetica}{46.1}}}} & \multicolumn{1}{>{\raggedright}m{\dimexpr 0.75in+0\tabcolsep}}{\textcolor[HTML]{000000}{\fontsize{11}{11}\selectfont{\global\setmainfont{Helvetica}{35.3}}}} \\





\multicolumn{1}{>{\raggedleft}m{\dimexpr 0.75in+0\tabcolsep}}{\textcolor[HTML]{000000}{\fontsize{11}{11}\selectfont{\global\setmainfont{Helvetica}{2,008}}}} & \multicolumn{1}{>{\raggedright}m{\dimexpr 0.75in+0\tabcolsep}}{\textcolor[HTML]{000000}{\fontsize{11}{11}\selectfont{\global\setmainfont{Helvetica}{115.2}}}} & \multicolumn{1}{>{\raggedright}m{\dimexpr 0.75in+0\tabcolsep}}{\textcolor[HTML]{000000}{\fontsize{11}{11}\selectfont{\global\setmainfont{Helvetica}{29.2}}}} & \multicolumn{1}{>{\raggedright}m{\dimexpr 0.75in+0\tabcolsep}}{\textcolor[HTML]{000000}{\fontsize{11}{11}\selectfont{\global\setmainfont{Helvetica}{36.1}}}} & \multicolumn{1}{>{\raggedright}m{\dimexpr 0.75in+0\tabcolsep}}{\textcolor[HTML]{000000}{\fontsize{11}{11}\selectfont{\global\setmainfont{Helvetica}{37.5}}}} \\





\multicolumn{1}{>{\raggedleft}m{\dimexpr 0.75in+0\tabcolsep}}{\textcolor[HTML]{000000}{\fontsize{11}{11}\selectfont{\global\setmainfont{Helvetica}{2,009}}}} & \multicolumn{1}{>{\raggedright}m{\dimexpr 0.75in+0\tabcolsep}}{\textcolor[HTML]{000000}{\fontsize{11}{11}\selectfont{\global\setmainfont{Helvetica}{26.6}}}} & \multicolumn{1}{>{\raggedright}m{\dimexpr 0.75in+0\tabcolsep}}{\textcolor[HTML]{000000}{\fontsize{11}{11}\selectfont{\global\setmainfont{Helvetica}{2.3}}}} & \multicolumn{1}{>{\raggedright}m{\dimexpr 0.75in+0\tabcolsep}}{\textcolor[HTML]{000000}{\fontsize{11}{11}\selectfont{\global\setmainfont{Helvetica}{46.6}}}} & \multicolumn{1}{>{\raggedright}m{\dimexpr 0.75in+0\tabcolsep}}{\textcolor[HTML]{000000}{\fontsize{11}{11}\selectfont{\global\setmainfont{Helvetica}{59.8}}}} \\





\multicolumn{1}{>{\raggedleft}m{\dimexpr 0.75in+0\tabcolsep}}{\textcolor[HTML]{000000}{\fontsize{11}{11}\selectfont{\global\setmainfont{Helvetica}{2,010}}}} & \multicolumn{1}{>{\raggedright}m{\dimexpr 0.75in+0\tabcolsep}}{\textcolor[HTML]{000000}{\fontsize{11}{11}\selectfont{\global\setmainfont{Helvetica}{44.6}}}} & \multicolumn{1}{>{\raggedright}m{\dimexpr 0.75in+0\tabcolsep}}{\textcolor[HTML]{000000}{\fontsize{11}{11}\selectfont{\global\setmainfont{Helvetica}{9}}}} & \multicolumn{1}{>{\raggedright}m{\dimexpr 0.75in+0\tabcolsep}}{\textcolor[HTML]{000000}{\fontsize{11}{11}\selectfont{\global\setmainfont{Helvetica}{35.3}}}} & \multicolumn{1}{>{\raggedright}m{\dimexpr 0.75in+0\tabcolsep}}{\textcolor[HTML]{000000}{\fontsize{11}{11}\selectfont{\global\setmainfont{Helvetica}{17.5}}}} \\





\multicolumn{1}{>{\raggedleft}m{\dimexpr 0.75in+0\tabcolsep}}{\textcolor[HTML]{000000}{\fontsize{11}{11}\selectfont{\global\setmainfont{Helvetica}{2,011}}}} & \multicolumn{1}{>{\raggedright}m{\dimexpr 0.75in+0\tabcolsep}}{\textcolor[HTML]{000000}{\fontsize{11}{11}\selectfont{\global\setmainfont{Helvetica}{38.4}}}} & \multicolumn{1}{>{\raggedright}m{\dimexpr 0.75in+0\tabcolsep}}{\textcolor[HTML]{000000}{\fontsize{11}{11}\selectfont{\global\setmainfont{Helvetica}{0}}}} & \multicolumn{1}{>{\raggedright}m{\dimexpr 0.75in+0\tabcolsep}}{\textcolor[HTML]{000000}{\fontsize{11}{11}\selectfont{\global\setmainfont{Helvetica}{79.9}}}} & \multicolumn{1}{>{\raggedright}m{\dimexpr 0.75in+0\tabcolsep}}{\textcolor[HTML]{000000}{\fontsize{11}{11}\selectfont{\global\setmainfont{Helvetica}{19.5}}}} \\





\multicolumn{1}{>{\raggedleft}m{\dimexpr 0.75in+0\tabcolsep}}{\textcolor[HTML]{000000}{\fontsize{11}{11}\selectfont{\global\setmainfont{Helvetica}{2,012}}}} & \multicolumn{1}{>{\raggedright}m{\dimexpr 0.75in+0\tabcolsep}}{\textcolor[HTML]{000000}{\fontsize{11}{11}\selectfont{\global\setmainfont{Helvetica}{79.2}}}} & \multicolumn{1}{>{\raggedright}m{\dimexpr 0.75in+0\tabcolsep}}{\textcolor[HTML]{000000}{\fontsize{11}{11}\selectfont{\global\setmainfont{Helvetica}{0}}}} & \multicolumn{1}{>{\raggedright}m{\dimexpr 0.75in+0\tabcolsep}}{\textcolor[HTML]{000000}{\fontsize{11}{11}\selectfont{\global\setmainfont{Helvetica}{85.1}}}} & \multicolumn{1}{>{\raggedright}m{\dimexpr 0.75in+0\tabcolsep}}{\textcolor[HTML]{000000}{\fontsize{11}{11}\selectfont{\global\setmainfont{Helvetica}{17.1}}}} \\





\multicolumn{1}{>{\raggedleft}m{\dimexpr 0.75in+0\tabcolsep}}{\textcolor[HTML]{000000}{\fontsize{11}{11}\selectfont{\global\setmainfont{Helvetica}{2,013}}}} & \multicolumn{1}{>{\raggedright}m{\dimexpr 0.75in+0\tabcolsep}}{\textcolor[HTML]{000000}{\fontsize{11}{11}\selectfont{\global\setmainfont{Helvetica}{31.2}}}} & \multicolumn{1}{>{\raggedright}m{\dimexpr 0.75in+0\tabcolsep}}{\textcolor[HTML]{000000}{\fontsize{11}{11}\selectfont{\global\setmainfont{Helvetica}{0}}}} & \multicolumn{1}{>{\raggedright}m{\dimexpr 0.75in+0\tabcolsep}}{\textcolor[HTML]{000000}{\fontsize{11}{11}\selectfont{\global\setmainfont{Helvetica}{115.1}}}} & \multicolumn{1}{>{\raggedright}m{\dimexpr 0.75in+0\tabcolsep}}{\textcolor[HTML]{000000}{\fontsize{11}{11}\selectfont{\global\setmainfont{Helvetica}{29.2}}}} \\





\multicolumn{1}{>{\raggedleft}m{\dimexpr 0.75in+0\tabcolsep}}{\textcolor[HTML]{000000}{\fontsize{11}{11}\selectfont{\global\setmainfont{Helvetica}{2,014}}}} & \multicolumn{1}{>{\raggedright}m{\dimexpr 0.75in+0\tabcolsep}}{\textcolor[HTML]{000000}{\fontsize{11}{11}\selectfont{\global\setmainfont{Helvetica}{56.2}}}} & \multicolumn{1}{>{\raggedright}m{\dimexpr 0.75in+0\tabcolsep}}{\textcolor[HTML]{000000}{\fontsize{11}{11}\selectfont{\global\setmainfont{Helvetica}{0}}}} & \multicolumn{1}{>{\raggedright}m{\dimexpr 0.75in+0\tabcolsep}}{\textcolor[HTML]{000000}{\fontsize{11}{11}\selectfont{\global\setmainfont{Helvetica}{250.1}}}} & \multicolumn{1}{>{\raggedright}m{\dimexpr 0.75in+0\tabcolsep}}{\textcolor[HTML]{000000}{\fontsize{11}{11}\selectfont{\global\setmainfont{Helvetica}{35.9}}}} \\

\ascline{1.5pt}{666666}{1-5}



\end{longtable*}



\arrayrulecolor[HTML]{000000}

\global\setlength{\arrayrulewidth}{\Oldarrayrulewidth}

\global\setlength{\tabcolsep}{\Oldtabcolsep}

\renewcommand*{\arraystretch}{1}

::: :::

:::

\newpage{}

\newpage{}

\subsection{Figures}\label{figures}

\phantomsection\label{refs}
\begin{CSLReferences}{1}{0}
\bibitem[\citeproctext]{ref-adams_diet_1987}
Adams, PB. 1987. {``The Diet of Widow Rockfish {Sebastes} Entomelas in
Northern {California}.''} \emph{NOAA Tech. Rep. NMFS} 48: 37--41.

\bibitem[\citeproctext]{ref-bradburn_2003_2011}
Bradburn, Mark James, Aimee A Keller, and Beth Helene Horness. 2011.
{``The 2003 to 2008 {US} {West} {Coast} Bottom Trawl Surveys of
Groundfish Resources Off {Washington}, {Oregon}, and {California}:
Estimates of Distribution, Abundance, Length, and Age Composition.''}

\bibitem[\citeproctext]{ref-dick_modeling_2009}
Dick, Edward Joseph. 2009. \emph{Modeling the Reproductive Potential of
Rockfishes ({Sebastes} Spp.)}. University of California, Santa Cruz.

\bibitem[\citeproctext]{ref-douglas_species_1998}
Douglas, David A. 1998. {``Species Composition of Rockfish in Catches by
{Oregon} Trawlers, 1963-93.''}

\bibitem[\citeproctext]{ref-hamel_method_2014}
Hamel, Owen S. 2014. {``A Method for Calculating a Meta-Analytical Prior
for the Natural Mortality Rate Using Multiple Life History
Correlates.''} \emph{ICES Journal of Marine Science} 72 (1): 62--69.

\bibitem[\citeproctext]{ref-he_status_2011}
He, Xi, Donald Pearson, E. J. Dick, John Field, Stephen Ralston, and
Alec MacCall. 2011. {``Status of the Widow Rockfish Resource in 2011.''}
Portland, OR: Pacific Fishery Management Council.

\bibitem[\citeproctext]{ref-jones_oregon_1960}
Jones, Walter G, and George Y Harry Jr. 1960. {``The {Oregon} Trawl
Fishery for Mink Food 1948-1957.''} \emph{Fish Commission of Oregon
Research Briefs} 8: 14--30.

\bibitem[\citeproctext]{ref-punt_quantifying_2008}
Punt, André E, David C Smith, Kyne KrusicGolub, and Simon Robertson.
2008. {``Quantifying Age-Reading Error for Use in Fisheries Stock
Assessments, with Application to Species in {Australia}'s Southern and
Eastern Scalefish and Shark Fishery.''} \emph{Canadian Journal of
Fisheries and Aquatic Sciences} 65 (9): 1991--2005.

\bibitem[\citeproctext]{ref-rogers_species_2003}
Rogers, Jean Beyer. 2003. {``Species Allocation of {Sebastes} and
{Sebastolobus} Sp. Caught by Foreign Countries from 1965 Through 1976
Off {Washington}, {Oregon}, and {California}, {USA}.''}

\bibitem[\citeproctext]{ref-stanley_estimation_2000}
Stanley, RD, R Kieser, K Cooke, AM Surry, and B Mose. 2000.
{``Estimation of a Widow Rockfish ({Sebastes} Entomelas) Shoal Off
{British} {Columbia}, {Canada} as a Joint Exercise Between Stock
Assessment Staff and the Fishing Industry.''} \emph{ICES Journal of
Marine Science} 57 (4): 1035--49.

\end{CSLReferences}




\end{document}
