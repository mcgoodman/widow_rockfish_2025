% Options for packages loaded elsewhere
\PassOptionsToPackage{unicode}{hyperref}
\PassOptionsToPackage{hyphens}{url}
\PassOptionsToPackage{dvipsnames,svgnames,x11names}{xcolor}
%
\documentclass[
]{scrartcl}

\usepackage{amsmath,amssymb}
\usepackage{iftex}
\ifPDFTeX
  \usepackage[T1]{fontenc}
  \usepackage[utf8]{inputenc}
  \usepackage{textcomp} % provide euro and other symbols
\else % if luatex or xetex
  \usepackage{unicode-math}
  \defaultfontfeatures{Scale=MatchLowercase}
  \defaultfontfeatures[\rmfamily]{Ligatures=TeX,Scale=1}
\fi
\usepackage{lmodern}
\ifPDFTeX\else  
    % xetex/luatex font selection
\fi
% Use upquote if available, for straight quotes in verbatim environments
\IfFileExists{upquote.sty}{\usepackage{upquote}}{}
\IfFileExists{microtype.sty}{% use microtype if available
  \usepackage[]{microtype}
  \UseMicrotypeSet[protrusion]{basicmath} % disable protrusion for tt fonts
}{}
\makeatletter
\@ifundefined{KOMAClassName}{% if non-KOMA class
  \IfFileExists{parskip.sty}{%
    \usepackage{parskip}
  }{% else
    \setlength{\parindent}{0pt}
    \setlength{\parskip}{6pt plus 2pt minus 1pt}}
}{% if KOMA class
  \KOMAoptions{parskip=half}}
\makeatother
\usepackage{xcolor}
\setlength{\emergencystretch}{3em} % prevent overfull lines
\setcounter{secnumdepth}{5}
% Make \paragraph and \subparagraph free-standing
\makeatletter
\ifx\paragraph\undefined\else
  \let\oldparagraph\paragraph
  \renewcommand{\paragraph}{
    \@ifstar
      \xxxParagraphStar
      \xxxParagraphNoStar
  }
  \newcommand{\xxxParagraphStar}[1]{\oldparagraph*{#1}\mbox{}}
  \newcommand{\xxxParagraphNoStar}[1]{\oldparagraph{#1}\mbox{}}
\fi
\ifx\subparagraph\undefined\else
  \let\oldsubparagraph\subparagraph
  \renewcommand{\subparagraph}{
    \@ifstar
      \xxxSubParagraphStar
      \xxxSubParagraphNoStar
  }
  \newcommand{\xxxSubParagraphStar}[1]{\oldsubparagraph*{#1}\mbox{}}
  \newcommand{\xxxSubParagraphNoStar}[1]{\oldsubparagraph{#1}\mbox{}}
\fi
\makeatother


\providecommand{\tightlist}{%
  \setlength{\itemsep}{0pt}\setlength{\parskip}{0pt}}\usepackage{longtable,booktabs,array}
\usepackage{calc} % for calculating minipage widths
% Correct order of tables after \paragraph or \subparagraph
\usepackage{etoolbox}
\makeatletter
\patchcmd\longtable{\par}{\if@noskipsec\mbox{}\fi\par}{}{}
\makeatother
% Allow footnotes in longtable head/foot
\IfFileExists{footnotehyper.sty}{\usepackage{footnotehyper}}{\usepackage{footnote}}
\makesavenoteenv{longtable}
\usepackage{graphicx}
\makeatletter
\def\maxwidth{\ifdim\Gin@nat@width>\linewidth\linewidth\else\Gin@nat@width\fi}
\def\maxheight{\ifdim\Gin@nat@height>\textheight\textheight\else\Gin@nat@height\fi}
\makeatother
% Scale images if necessary, so that they will not overflow the page
% margins by default, and it is still possible to overwrite the defaults
% using explicit options in \includegraphics[width, height, ...]{}
\setkeys{Gin}{width=\maxwidth,height=\maxheight,keepaspectratio}
% Set default figure placement to htbp
\makeatletter
\def\fps@figure{htbp}
\makeatother
% definitions for citeproc citations
\NewDocumentCommand\citeproctext{}{}
\NewDocumentCommand\citeproc{mm}{%
  \begingroup\def\citeproctext{#2}\cite{#1}\endgroup}
\makeatletter
 % allow citations to break across lines
 \let\@cite@ofmt\@firstofone
 % avoid brackets around text for \cite:
 \def\@biblabel#1{}
 \def\@cite#1#2{{#1\if@tempswa , #2\fi}}
\makeatother
\newlength{\cslhangindent}
\setlength{\cslhangindent}{1.5em}
\newlength{\csllabelwidth}
\setlength{\csllabelwidth}{3em}
\newenvironment{CSLReferences}[2] % #1 hanging-indent, #2 entry-spacing
 {\begin{list}{}{%
  \setlength{\itemindent}{0pt}
  \setlength{\leftmargin}{0pt}
  \setlength{\parsep}{0pt}
  % turn on hanging indent if param 1 is 1
  \ifodd #1
   \setlength{\leftmargin}{\cslhangindent}
   \setlength{\itemindent}{-1\cslhangindent}
  \fi
  % set entry spacing
  \setlength{\itemsep}{#2\baselineskip}}}
 {\end{list}}
\usepackage{calc}
\newcommand{\CSLBlock}[1]{\hfill\break\parbox[t]{\linewidth}{\strut\ignorespaces#1\strut}}
\newcommand{\CSLLeftMargin}[1]{\parbox[t]{\csllabelwidth}{\strut#1\strut}}
\newcommand{\CSLRightInline}[1]{\parbox[t]{\linewidth - \csllabelwidth}{\strut#1\strut}}
\newcommand{\CSLIndent}[1]{\hspace{\cslhangindent}#1}

\usepackage{hyphenat}
\usepackage{graphicx}
% and their extensions so you won't have to specify these with
 % every instance of \includegraphics
 \usepackage{pdfcomment}
\DeclareGraphicsExtensions{.pdf,.jpeg,.png}
\usepackage{wallpaper} % for the background image on title page
\usepackage{geometry}
% set font

% added by Ross
% % set font - - depends upon the driver
% \ifPDFTeX
%  %% only want this in body section headings and ToC, using \sf
%  \def\sfdefault{phv}% Helvetica instead of its clone Arial
%  \renewcommand{\sectfont}{\normalcolor
%   \def\bfdefault{bc}% bold condensed; i.e., narrow
%   \maybesffamily \bfseries }%% uses uhvb8ac
% % \def\sfdefault{lmss}% Latin Modern replaces Arial
% % \renewcommand{\sectfont}{\normalcolor
%  % \fontseries{sbc}\fontfamily{lmss}\selectfont }%% uses lmssdc10
% \else

\usepackage{fontspec}
\setsansfont[Ligatures=TeX]{Arial Narrow}

% added by Ross
%\fi
%\usepackage[scaled=0.9]{helvet}% needed later to replace Arial Narrow
\usepackage[headsepline=0.005pt:,footsepline=0.005pt:,plainfootsepline,automark]{scrlayer-scrpage}
\clearpairofpagestyles
\ohead[]{\headmark} \cofoot[\pagemark]{\pagemark}
\ModifyLayer[addvoffset=-.6ex]{scrheadings.foot.above.line}
\ModifyLayer[addvoffset=-.6ex]{plain.scrheadings.foot.above.line}
\setkomafont{pageheadfoot}{\small}
% Landscape tables and figures
\usepackage{pdflscape}
\newcommand{\blandscape}{\begin{landscape}}
\newcommand{\elandscape}{\end{landscape}}

% Acronyms
\usepackage[acronym]{glossaries}
\glsdisablehyper
\loadglsentries{sa4ss_glossaries.tex}
\lohead{SPECIES assessment 2025}
\makeatletter
\@ifpackageloaded{caption}{}{\usepackage{caption}}
\AtBeginDocument{%
\ifdefined\contentsname
  \renewcommand*\contentsname{Table of contents}
\else
  \newcommand\contentsname{Table of contents}
\fi
\ifdefined\listfigurename
  \renewcommand*\listfigurename{List of Figures}
\else
  \newcommand\listfigurename{List of Figures}
\fi
\ifdefined\listtablename
  \renewcommand*\listtablename{List of Tables}
\else
  \newcommand\listtablename{List of Tables}
\fi
\ifdefined\figurename
  \renewcommand*\figurename{Figure}
\else
  \newcommand\figurename{Figure}
\fi
\ifdefined\tablename
  \renewcommand*\tablename{Table}
\else
  \newcommand\tablename{Table}
\fi
}
\@ifpackageloaded{float}{}{\usepackage{float}}
\floatstyle{ruled}
\@ifundefined{c@chapter}{\newfloat{codelisting}{h}{lop}}{\newfloat{codelisting}{h}{lop}[chapter]}
\floatname{codelisting}{Listing}
\newcommand*\listoflistings{\listof{codelisting}{List of Listings}}
\makeatother
\makeatletter
\makeatother
\makeatletter
\@ifpackageloaded{caption}{}{\usepackage{caption}}
\@ifpackageloaded{subcaption}{}{\usepackage{subcaption}}
\makeatother

\ifLuaTeX
\usepackage[bidi=basic]{babel}
\else
\usepackage[bidi=default]{babel}
\fi
\babelprovide[main,import]{english}
% get rid of language-specific shorthands (see #6817):
\let\LanguageShortHands\languageshorthands
\def\languageshorthands#1{}
\ifLuaTeX
  \usepackage{selnolig}  % disable illegal ligatures
\fi
\usepackage{bookmark}

\IfFileExists{xurl.sty}{\usepackage{xurl}}{} % add URL line breaks if available
\urlstyle{same} % disable monospaced font for URLs
\hypersetup{
  pdftitle={Status of SPECIES off the U.S. West Coast in 2025},
  pdfauthor={Michael Kinneen},
  pdflang={en},
  colorlinks=true,
  linkcolor={blue},
  filecolor={Maroon},
  citecolor={Blue},
  urlcolor={Blue},
  pdfcreator={LaTeX via pandoc}}


\title{Status of SPECIES off the U.S. West Coast in 2025}
\author{Michael Kinneen}
\date{2025-05-03}

\begin{document}
  \begin{titlepage}
  % This is a combination of Pandoc templating and LaTeX
  % Pandoc templating https://pandoc.org/MANUAL.html#templates
  % See the README for help

  \newgeometry{top=2in,bottom=1in,right=1in,left=1in}
  \begin{minipage}[b][\textheight][s]{\textwidth}
  % Ross would've subbed lines 6, 8 with these lines:
  %\newgeometry{top=2in,bottom=1in,right=1in,left=1in}%
  %\noindent  %\tracingall
  %\begin{minipage}[b][\textheight][s]{.975\textwidth}%% RRM: avoid Overfull box


  \raggedright

  % \includegraphics[width=2cm]{NOAA_Transparent_Logo.png}

  % background image


  % Title and subtitle
  {\huge\bfseries\nohyphens{Status of SPECIES off the U.S. West Coast in
  2025}}\\[1\baselineskip]
  % Ross would change the end of the above line to the following because \par must come before the group closes and line-depth reverts.
  % }\par}%\\[1\baselineskip]



  \vspace{1\baselineskip}
  % Ross would change this to 2\baselineskip

  %%%%%% Cover image

  \vspace{1\baselineskip}

  % Authors
  % This hairy bit of code is just to get "and" between the last 2
  % authors. See below if you don't need that
  %
  {\large{Michael Kinneen}}%
  %
  {\textsuperscript{1}}%
  %


  % This is how to do it if you don't need the "and"

  %%%%%% Affiliations
  \vspace{2\baselineskip}

  \hangindent=1em
  \hangafter=1
  % Ross would change the above line to:
  % \hangafter=1\relax
  %
  {1}.~{NOAA Fisheries Northwest Fisheries Science Center}%
  %
  %
  % Ross recommends putting address on one line
  , %
  {2725 Montlake Boulevard East}%
  %


  %%%%%% Correspondence
  \vspace{1\baselineskip}


  %use \vfill instead to get the space to fill flexibly
  %\vspace{0.25\textheight} % Whitespace between the title block and the publisher

  \vfill


  % Whitespace between the title block and the tagline
  \vspace{1\baselineskip}

  %%%%%% Tagline at bottom
  % Ross says the tagline below could also be centered
  \includegraphics[alt={},width=2cm]{support_files/us_doc_logo.png}\newline % empty curly brackets without alt text is suitable for this logo because it's purely decorative/an "artifact"
  U.S. Department of Commerce\newline
  National Oceanic and Atmospheric Administration\newline
  National Marine Fisheries Service\newline
  Northwest Fisheries Science Center\newline

  \end{minipage}
  \restoregeometry
  \end{titlepage}

\renewcommand*\contentsname{Table of contents}
{
\hypersetup{linkcolor=}
\setcounter{tocdepth}{3}
\tableofcontents
}

\newpage{}

Please cite this publication as:

Michael Kinneen, Maurice Goodman, . (2025) Status of Widow Rockfish off
the U.S. West Coast in 2025. Pacific Fishery Management Council.
{[}XX{]} p.

\newpage{}

\pagenumbering{roman}
\setcounter{page}{1}

\renewcommand{\thetable}{\roman{table}}
\renewcommand{\thefigure}{\roman{figure}}

\section*{Disclaimer}\label{disclaimer}
\addcontentsline{toc}{section}{Disclaimer}

These materials do not constitute a formal publication and are for
information only. They are in a pre-review, pre-decisional state and
should not be formally cited or reproduced. They are to be considered
provisional and do not represent any determination or policy of NOAA or
the Department of Commerce.

\newpage{}

\section*{Executive Summary}\label{executive-summary}
\addcontentsline{toc}{section}{Executive Summary}

Checking to see if this works \gls{s-tri}

\subsection*{Stock}\label{stock}
\addcontentsline{toc}{subsection}{Stock}

\subsection*{Catches}\label{catches}
\addcontentsline{toc}{subsection}{Catches}

\subsection*{Data and Assessment}\label{data-and-assessment}
\addcontentsline{toc}{subsection}{Data and Assessment}

\subsection*{Stock biomass and
dynamics}\label{stock-biomass-and-dynamics}
\addcontentsline{toc}{subsection}{Stock biomass and dynamics}

\subsection*{Recruitment}\label{recruitment}
\addcontentsline{toc}{subsection}{Recruitment}

\subsection*{Exploitation status}\label{exploitation-status}
\addcontentsline{toc}{subsection}{Exploitation status}

\subsection*{Ecosystem considerations}\label{ecosystem-considerations}
\addcontentsline{toc}{subsection}{Ecosystem considerations}

\subsection*{Reference points}\label{reference-points}
\addcontentsline{toc}{subsection}{Reference points}

\subsection*{Management performance}\label{management-performance}
\addcontentsline{toc}{subsection}{Management performance}

\subsection*{Harvest projections}\label{harvest-projections}
\addcontentsline{toc}{subsection}{Harvest projections}

\subsection*{Decision table}\label{decision-table}
\addcontentsline{toc}{subsection}{Decision table}

\subsection*{Scientific uncertainty}\label{scientific-uncertainty}
\addcontentsline{toc}{subsection}{Scientific uncertainty}

\subsection*{Research and data needs}\label{research-and-data-needs}
\addcontentsline{toc}{subsection}{Research and data needs}

\subsection*{Rebuilding projections}\label{rebuilding-projections}
\addcontentsline{toc}{subsection}{Rebuilding projections}

\newpage{}

\setlength{\parskip}{5mm plus1mm minus1mm}
\pagenumbering{arabic}
\setcounter{page}{1}
\setcounter{section}{0}
\renewcommand{\thefigure}{\arabic{figure}}
\renewcommand{\thetable}{\arabic{table}}
\setcounter{table}{0}
\setcounter{figure}{0}

\section{Introduction}\label{introduction}

\subsection{Life History}\label{life-history}

Refer to the most recent full assessment for additional information.

\subsection{Ecosystem considerations}\label{ecosystem-considerations-1}

Refer to the most recent full assessment for additional information.

\subsection{Fishery description}\label{fishery-description}

Refer to the most recent full assessment for additional information.

\subsection{Management History}\label{management-history}

Refer to the most recent full assessment for additional information.

\subsection{Management performance}\label{management-performance-1}

\subsection{Fisheries off Canada and
Alaska}\label{fisheries-off-canada-and-alaska}

Refer to the most recent full assessment for additional information.

\newpage{}

\section{Data}\label{data}

Many sources of data were available for this assessment, including
indices of abundance (Table 5), length observations, and age
observations from fishery-dependent and fishery-independent sources.

\subsection{Fishery-independent data}\label{fishery-independent-data}

Data from three fishery-independent surveys were used in this
assessment: 1) the SWFSC and NWFSC/PWCC Midwater Trawl Survey
(hereafter, ``juvenile survey''); 2) the Alaska Fisheries Science Center
(AFSC)/NWFSC Triennial Shelf Trawl Survey (hereafter, ``triennial
survey''); and 3) the NWFSC West Coast Groundfish Bottom Trawl Survey
(hereafter, ``WCGBTS''). These surveys employed different designs and
sampling methodologies, were conducted during different years and time
periods within years, and included coverage over different areas of the
coast. In some instances, the survey frequency, depths, and geographic
areas covered were not internally consistent within surveys. A brief
description of each survey is provided below. Strata were defined by
latitude and depth to analyze the catch-rates, length compositions, and
age compositions using stratified random sampling theory (Table 6 \&
Table 7). The latitude and depth breaks were chosen based on the design
of the survey as well as by looking at biological patterns in relation
to latitude and depth. Indices of abundance for all of the surveys were
derived using model based approaches described below.

\subsubsection{\texorpdfstring{\acrlong{s-tri}}{}}\label{section}

The \gls{s-tri} was first conducted by the \gls{afsc} in 1977, and the
survey continued until 2004 (Weinberg et al. 2002). Its basic design was
a series of equally-spaced east-to-west transects across the
continential shelf from which searches for tows in a specific depth
range were initiated. The survey design changed slightly over time. In
general, all of the surveys were conducted in the mid summer through
early fall. The 1977 survey was conducted from early July through late
September. The surveys from 1980 through 1989 were conducted from the
middle of July to late September. The 1992 survey was conducted from the
middle of July through early October. The 1995 survey was conducted from
early June through late August. The 1998 survey was conducted from early
June through early August. Finally, the 2001 and 2004 surveys were
conducted from May to July.

Haul depths ranged from 91--457 m during the 1977 survey. Due to haul
performance issues and truncated sampling with respect to depth, the
data from 1977 were omitted from this analysis. The surveys in 1980,
1983, and 1986 covered the U.S. West Coast south to 36.8\textdegree N
latitude and a depth range of 55--366 m. The surveys in 1989 and 1992
covered the same depth range but extended the southern range to
34.5\textdegree N (near Point Conception). From 1995 through 2004, the
surveys covered the depth range 55--500 m and surveyed south to
34.5\textdegree N. In 2004, the final year of the \gls{s-tri} series,
\gls{nwfsc} \gls{fram} conducted the survey following similar protocols
to earlier years.

\subsubsection{\texorpdfstring{\acrlong{s-wcgbt}}{}}\label{section-1}

The \gls{s-wcgbt} is based on a random-grid design; covering the coastal
waters from a depth of 55--1,280 m (Bradburn, Keller, and Horness 2011).
This design generally uses four industry-chartered vessels per year
assigned to a roughly equal number of randomly selected grid cells and
divided into two `passes' of the coast. Two vessels fish from north to
south during each pass between late May to early October. There were
only two vessels used in 2019 and three in 2013, with one of the three
that year unable to complete its survey pass due to a government
shutdown. No survey occurred in 2020 due to \gls{covid}. This design
therefore incorporates both vessel-to-vessel differences in
catchability, as well as variance associated with selecting a relatively
small number (approximately 700) of possible cells from a very large set
of possible cells spread from the Mexican to the Canadian borders. Note
that the Survey is not permitted to access the \glspl{cca} in Southern
California.

Widow Rockfish are not commonly caught in the WCGBTS. Higher catch rates
occur north of 40° N latitude and catches are rare south of 36° N
latitude (Figure 11). Few large fish are found shallower than 100 m and
few small fish are found in the deeper water of the slope. There is no
clear trend in length with latitude other than smaller fish tend to
occur south of approximately 36° N latitude, and there appears to be
some very small fish found near 39° N latitude.

An index was created using spatiotemporal species distribution modeling
via the sdmTMB package (citation). This reflects an updated approach
compared to the 2015 assessment (non-spatial delta-GLMM) and the 2019
update assessment (VAST delta-lognormal model). The sdmTMB index
estimates spatial and spatiotemporal variation in encounter probability
and positive catch weight across the survey range using a 200 knot grid.
The positive catch weight model includes survey pass (`first' or
`second') to account for missing data, i.e., the incomplete second pass
of the 2013 survey. Spatiotemporal estimates of weight are then
converted into annual indices using XXX (something from sdmTMB??). Both
gamma and lognormal error structures were tested for the positive catch
model. Both models converged (positive, definite Hessian matrix) but
predicted data from both models showed slightly right-heavy tails
compared to null expectations, with the gamma model having stronger
divergence.

The index estimate is relatively stable, with a slightly increasing
trend in recent years and a moderate peak in 2016. Overall, the
lognormal index estimates were more comparable to the 2019
spatiotemporal VAST-based index and seemed less influenced by potential
extreme catch events, particularly in 2013 and 2016 {[}is there a more
stats-y reason, e.g., AIC score{]}; for these reasons, the
delta-lognormal sdmTMB-based index was used for the base model in this
assessment. {[}The delta-lognormal mean value (XX) was slightly lower
than the means of the index values used in the 2015 assessment (2701.12)
and the 2019 update assessment (3301.765).{]} Comparisons of the
different error structures, design-based estimate and the VAST index
used in 2019 are in Figure XX.

\subsection{Fishery-dependent data}\label{fishery-dependent-data}

Widow Rockfish have been caught in trawl and hook-and-line fisheries
since the early part of the 20th century. Widow Rockfish are a desirable
rockfish and are not likely to be discarded for market reasons. However,
smaller Widow Rockfish are found at shallower depths and discarding
practices in the early 1900s are uncertain. Few Widow Rockfish have been
observed (relative to other gear types) in recreational, commercial pot,
and commercial shrimp fisheries, thus only trawl, net, and hook-and-line
landings were used in this assessment.

In data from the early 1980s, Widow Rockfish have had their own landing
category. California began in 1982, Oregon in 1984, and Washington in
1988. Estimates of historical landings of Widow Rockfish rely upon
species-composition sampling data from each period. The uncertainty in
species composition is greater in past years, with less systematic and
extensive sampling occurring prior to 1980. Consequently, the precision
with which landings of Widow Rockfish can be estimated likely decreases
for earlier years. A description of the methods used to determine the
historical and current landings is provided below

\subsubsection{Fishery length and age
data}\label{fishery-length-and-age-data}

Biological data from commercial fisheries that caught Widow Rockfish
were extracted from PacFIN (PSMFC) on July 3, 2019, from CALCOM on July
3, 2019 and from the NORPAC database on July 3, 2019. Lengths taken
during port sampling in California, Oregon, and Washington were used to
calculate length and age compositions. The data were classified into
bottom trawl, midwater trawl, hake trawl, net, and hook-and-line fleets

Table 10 shows the number of landings sampled and Table 11 shows the
number of lengths taken for each year, gear, and fleet from the three
states. Table 12 shows these numbers for the at-sea fleet.

Consistent with the 2015 assessment, length and age samples from PacFIN
and CALCOM were expanded up to the total landing then combined into
state-specific frequencies (Table 13). Expansion factors were calculated
in a way such that large expansions would not occur and based on ideas
first presented by Owen Hamel (pers. comm., NWFSC). First the expansion
factor (Ek) was the total catch weight (Wk) divided by the sample weight
(wk), and raised to 0.9 to account for non-homogeneity within a trip.
Then, expansion factors greater than 300 were capped (100 for net
fisheries) to reduce the influence of small samples (i.e., a few fish
representing a large catch). The predicted total numbers at length or
age weighted by landings for each state were added to create a
coast-wide length frequency. The effective sample sizes of the state
combined length frequencies were determined from the following formula,
which has been used in previous Widow Rockfish assessments as well as
other west coast groundfish assessments.

\begin{longtable}[]{@{}
  >{\raggedright\arraybackslash}p{(\columnwidth - 2\tabcolsep) * \real{0.4948}}
  >{\raggedright\arraybackslash}p{(\columnwidth - 2\tabcolsep) * \real{0.5052}}@{}}
\toprule\noalign{}
\begin{minipage}[b]{\linewidth}\raggedright
Fishery Samples
\end{minipage} & \begin{minipage}[b]{\linewidth}\raggedright
Survey Samples
\end{minipage} \\
\midrule\noalign{}
\endhead
\bottomrule\noalign{}
\endlastfoot
\(\ N_{eff} = N_{sample} + 0.138N_{fish} , \frac{N_{fish}}{N_{sample}} < 44\)
&
\(\ N_{eff} = N_{sample} + 0.0707N_{fish} , \frac{N_{fish}}{N_{sample}} < 55\) \\
\(\ N_{eff} = 7.06N_{sample} , \frac{N_{fish}}{N_{sample}} \geq 44\) &
\(\ N_{eff} = 4.89N_{sample} , \frac{N_{fish}}{N_{sample}} \geq 55\) \\
\end{longtable}

This is slightly different than the sample size of 2.43 per haul for
rockfish that Stewart \& Hamel (2014) report.

Observed lengths were expanded to the tow from At-Sea Hake Observer
Program samples (NORPAC). Tows are typically well sampled, thus
expansion factors were not modified from what was calculated. Hake
fishery length compositions were created by combining shoreside and
at-sea length compositions, weighting by the catch from each sector. The
effective sample sizes for hake fishery length and age comps were
calculated using the above equations for the shoreside fleet and added
to the number of tows sampled from the at-sea fleet.

Expanded length compositions for bottom trawl, midwater trawl, hake
fisheries, net, and hook-and-line are shown in Figure 17 to Figure 21.
It is quickly apparent that all of these fisheries rarely land fish less
than 26 cm. All of the non-hake fleets show a strong cohort coming
though in the late 1970s and early 1980s, and then another cohort coming
through in the late 1980s. Sample sizes typically dropped off after
2000, except in the hake fishery where nearly every tow is sampled. Age
compositions for the five fleets are shown in Figure 22 and Figure 26.
Occasional cohorts appear to move through the population, indicating
that Widow Rockfish population dynamics may be characterized by episodic
recruitment events.

\subsubsection{Discards}\label{discards}

-- TO DO!!! ---

\[
D_{y,f} = \frac{d_{y,f}}{r_{y,f}} R_{y,f}
\]

\subsubsection{Biological data}\label{biological-data}

\paragraph{Weight-length relationship}\label{weight-length-relationship}

Weight-at-length data, which are the same used in the 2015 assessment,
were collected from fisheries sampling and by the Triennial and NWFSC
WCGBT Surveys, and were used to estimate a weight-length relationship
for Widow Rockfish (Figure 30). Weight-at-length was similar between
sources with the fishery samples showing a slightly smaller weight at
large sizes when compared to the survey data (Figure 31). WCGOP data
were not used because only small fish were sampled, the weight of these
small fish were typically less than from other sources (Figure 30), and
the curves fitted to only WCGOP data were unable to estimate the slope.
There were only 81 observations from the WCGOP data, which is a small
amount of data compared to everything available. However, these
observations may be useful to understand discards.

The weight-length relationship used in the 2011 assessment was similar
for males but predicted slightly heavier females at larger sizes than
the 2015 assessment (Figure 31). The following relationships between
weight and length for females and males were estimated for the 2015
assessment from all of the data combined and were used in the current
assessment:

\[
\text{Females: } \qquad weight = 1.7355 \times 10^{-5} \cdot Length^{2.9617}
\]

\[
\text{Males: } \qquad weight = 1.4824 \times 10^{-5} \cdot Length^{3.0047}
\]

where weight is measured in kilograms and length in cm. These
relationships were used in the assessment as fixed relationships.

\paragraph{Maturity schedule}\label{maturity-schedule}

Estimates of maturity used in this update were the same as the 2015
assessment. Estimates of maturity at length have been presented by Barss
\& Echeverria (1987), Echeverria (1987), and Love et al (1990). Barss \&
Echeverria (1987) supplied data collected from Oregon and California
commercial and recreational samples, which allowed us to estimate the
proportion mature-at-length and proportion mature-at-age for samples
from each state (Figure 32). As noted by Barss \& Echeverria (1987), the
samples from Oregon matured at older age and larger length. Estimates of
maturity-at-length from California reported by Barss \& Echeverria
(1987) are similar to estimates of length-at-50\%-mature from samples
collected in California reported by Echeverria (1987) and Love et al
(1990), although Barss \& Echeverria show the smallest
length-at-50\%-mature. To maintain some consistency with the 2011
assessment and to avoid any potential growth issues by area, the 2015
assessment used maturity-at-age data from the 2011 assessment, but used
the data provided by Barss \& Echeverria (1987) to estimate a new
maturity curve following a logistic function with the data from
California and Oregon equally weighted to avoid California dominating
the estimated relationship. This maturity-at-age curve falls between the
estimated California and Oregon maturity-at-age curves (Figure 32,
right), with the age-at-50\%-mature estimated at 5.47 and with a slope
of -0.7747 (as specified in SS). This logistic maturity-at-age curve was
used in the 2015 and 2019 update assessment except that maturity-at-age
for ages 2 and lower were set equal to zero (Table 19).

\paragraph{Fecundity}\label{fecundity}

Fecundity in rockfish is often not a linear function of weight, but
increases faster at larger weights (Dick 2009). Therefore, this
relationship is often accounted for in rockfish assessments by using
spawning output (numbers of eggs) to determine current status. Dick
(2009) did not find a significant relationship between the number of
eggs per gram of body weight and body weight for Widow Rockfish.
Therefore, spawning output was assumed to be proportional to weight,
which is the same as spawning biomass, and is reported here.

\paragraph{Natural Mortality}\label{natural-mortality}

Natural mortality used in this update differed from the 2015 assessment.
Natural mortality (M) is a parameter that is often highly uncertain in
fish stocks. Past assessments of Widow Rockfish assumed constant natural
mortality of 0.125 yr-1 or 0.15 yr-1. The 2011 assessment estimated M
with a prior developed by Owen Hamel (NWFSC, pers. comm.) using methods
described in Hamel (2014). This prior was based on a maximum age of 44
and 40 for females and males, respectively, a mean temperature of 8
degrees Celsius (about 150m deep off of Oregon), and a gonadosomatic
index of 9.99\% and 1.86\% for females and males, respectively (Love et
al 1990). The sex-specific lognormal priors for M have medians 11 of
0.124 yr-1 and 0.129 yr-1 for females and males, respectively, and a
coefficient of variation (CV) of 30.7\% for each sex. In 2015,
discussions with Owen Hamel (NWFSC) led to the development of a new
prior based solely on maximum age to use when estimating M. Using all of
the available age data, a maximum age of 54 was determined for both
females and males, although it has been rare to observe Widow Rockfish
older than about 45 years old (Figure 33). This resulted in a prior with
a much smaller median (0.0810 or -2.513284 in log space) and a larger
standard deviation in log space (0.523694). For the update assessment,
an updated meta-anaysis resulted in a prior with a slightly smaller
median than the 2015 assessment (0.10 or -2.30 in log space) and a
smaller standard deviation in log space (0.438). Figure 34 shows that
these prior distributions are wide and not highly informative.

\paragraph{Length-at-age}\label{length-at-age}

Estimates of length-at-age used in this update were the same as the 2015
assessment. Two different labs have aged the majority of processed
otoliths for Widow Rockfish. The SWFSC has been aging Widow Rockfish
otoliths for many years, including all of the fishery data prior to 2011
and otoliths collected from the NWFSC WCGBT survey in 2009 and 2010. The
Cooperative Ageing Project (CAP) in Newport, Oregon aged 1,100 otoliths
from the NWFSC WCGBT survey, 2,026 otoliths provided by ASHOP, and 3,467
otoliths collected by port samplers. All of the commercial fishery
samples were collected in the years 2011--2014. In total, there are
105,814 paired age and length observations ranging from 1978 to 2014.
Figure 35 shows the lengths and ages for all years and all data as well
as predicted von Bertalanffy fits to the data. Females grow larger than
males and sex specific growth parameters were estimated at the following
values:

\[
\text{Females: } \qquad  L_\infty = 50.34,\quad k=0.15, \quad t_0 = -2.22
\]

\[
\text{Males: } \qquad  L_\infty = 44.19,\quad k=0.21, \quad t_0 = -1.78
\]

The data from each source (ASHOP, port sampling/BDS, Triennial survey,
and NWFSC survey) are shown in Figure 36 with fitted von Bertalanffy
lines. All of these sources are quite similar, especially observations
from ASHOP and the NWFSC survey. The standard deviation (SD) and
coefficient of variation (CV) of length-at-age are shown in Figure 37.
Modelling the CV as a function of predicted length-at-age appears to be
somewhat linear from a value just over 0.1 at small lengths and slightly
less than 0.045 at larger lengths. However, variance in length- at-age
was estimated separately in stock-synthesis.

\paragraph{Sex ratios}\label{sex-ratios}

Females tend to grow larger than males and it is expected that the
proportion of females approaches one at large lengths and is less than
0.5 at intermediate lengths. Figure 38 shows that the proportion of
females at length from survey data is approximately 50\% until
approximately 34 cm, when the proportion of females drops below 50\%. At
lengths larger than 46 cm, the proportion of females increases rapidly
to one, suggesting that few males grow larger than 50 cm

\paragraph{Ageing bias and
imprecision}\label{ageing-bias-and-imprecision}

Uncertainty surrounding the ageing-error process for widow rockfish used
in the 2015 assessment was incorporated by estimating ageing error by
age. No changes were made from the 2015 assessment for the update.
Age-composition data used in the model were from break-and-burn and
surface reads and were aged by the Cooperative Ageing Project (CAP) in
Newport, Oregon and the SWFSC in Santa Cruz, California. 12
Break-and-burn double reads of 1788 otoliths were performed by both the
CAP and the SWFSC lab combined. Additionally, 100 otoliths were read
both by surface and break-and-burn methods. An ageing error estimate was
made based on these double reads using a computational tool specifically
developed for estimating ageing error (Punt et al.~2008), and using
release 1.0.0 of the R package nwfscAgeingError (Thorson et al.~2012)
for input and output diagnostics, publicly available at:
https://github.com/nwfsc- assess/nwfscAgeingError. The maximum aged fish
read by the surface reading method was 10 years and the cross otolith
reads between the surface and break-and-burn ageing methods showed
limited variation. Therefore, a unique ageing error was not created for
surface read otoliths. A non-linear standard error was estimated by age
where there is more variability in the estimated age of older fish was
estimated for each reading lab (Table 20 and Figure 39).

\subsubsection{Abundance indices}\label{abundance-indices}

\subsection{Environmental and ecosystem
data}\label{environmental-and-ecosystem-data}

\newpage{}

\section{Assessment model}\label{assessment-model}

An age-structured stock assessment model was used to predict the biomass
trajectory of Widow Rockfish with an approach of balancing parsimony
with complexity. This allowed for the determination of general trends in
the biomass over time without introducing extraneous data partitions
that explain little additional variation. The assessment followed the
same model structure as the 2015 base assessment.

\subsection{History of modeling
approaches}\label{history-of-modeling-approaches}

Refer to the most recent full assessment for additional information.

\subsection{Responses to SSC Groundfish Subcommittee
requests}\label{responses-to-ssc-groundfish-subcommittee-requests}

To be completed after review.

\subsection{Model Structure and
Assumptions}\label{model-structure-and-assumptions}

\subsubsection{Model Changes from the Last
Assessment}\label{model-changes-from-the-last-assessment}

\subsubsection{Modeling Platform and
Structure}\label{modeling-platform-and-structure}

\subsubsection{Model Parameters}\label{model-parameters}

\subsubsection{Key Assumptions and Structural
Choices}\label{key-assumptions-and-structural-choices}

Refer to the most recent full assessment for additional information.

\subsection{Base Model Results}\label{base-model-results}

\subsubsection{Parameter Estimates}\label{parameter-estimates}

\subsubsection{Fits to the Data}\label{fits-to-the-data}

\subsubsection{Population Trajectory}\label{population-trajectory}

\subsection{Model Diagnostics}\label{model-diagnostics}

Three types of uncertainty are presented for the assessment of Widow
Rockfish. First, uncertainty in the parameter estimates was determined
using approximate asymptotic estimates of the standard error. These
estimates were based on the maximum likelihood theory that the inverse
of the Hessian matrix (the second derivative of the log-likelihood
function with respect to the parameter vector) approaches the true
uncertainty of the parameter estimates as the sample size approaches
infinity. This approach takes into account the uncertainty in the data
and supplies correlation estimates between parameters, but does not
capture possible skewness in the error distribution of the parameters
and may not accurately estimate the standard error in some cases (see
Stewart et al.~2013) UPDATE REF!!!!!!.

The second type of uncertainty that is presented is related to modeling
and structural error. This uncertainty cannot be captured in the base
model as it is related to errors in the assumptions used in specifying
the base model. Therefore, sensitivity analyses were conducted where
assumptions were modified to reveal the effect they have on the model
results.

Lastly, a major axis of uncertainty was determined from a parameter or
structural assumption that results in the greatest change in stock
status and advice, and projections were made for different states of
nature based upon that parameter or structural assumption.

\subsubsection{Convergence}\label{convergence}

\subsubsection{Parameter Uncertainty}\label{parameter-uncertainty}

Parameter estimates are shown in Table 22, Table 23, and Table 28 along
with approximate asymptotic standard errors. The only parameters with an
absolute value of correlation greater than 0.95 were the female and male
natural mortality parameters, which is expected. Estimates of key
derived quantities are given in Table 26 along with approximate 95\%
asymptotic confidence intervals. There is a reasonable amount of
uncertainty in the estimates of biomass. The confidence interval of the
2019 estimate of depletion is XX\%--XX\% and above the management target
of 40\% of the unfished spawning biomass.

\subsubsection{Sensitivity Analyses}\label{sensitivity-analyses}

\subsubsection{Retrospective Analysis}\label{retrospective-analysis}

\subsubsection{Likelihood Profiles and key
parameters}\label{likelihood-profiles-and-key-parameters}

Likelihood profiles were conducted for R0, steepness (even though it was
not estimated in the base case) and over male and female natural
mortality values simultaneously. These likelihood profiles were
conducted by fixing the parameter at specific values and removing the
prior on the parameter being profiled. Without the original prior
distribution the MLE estimates from the base case will likely be
different than the MLE in the likelihood profile, but this displays what
information the data have. There was some difficulty in achieving model
convergence for many parameterizations in the likelihood profile. In
some cases jittering was required.

As R0 increased, natural mortality also increased and the relative
spawning biomass in 2015 was less depleted (Table 31). There was
variable support for each likelihood component across the range of R0
evaluated. The total likelihood supported the estimated value (Table
31). Profiles are illustrated in Figure 68.

\subsection{Unresolved Problems and Major
Uncertainties}\label{unresolved-problems-and-major-uncertainties}

\newpage{}

\section{Management}\label{management}

\subsection{Reference Points}\label{reference-points-1}

\subsection{Harvest Projections and Decision
Tables}\label{harvest-projections-and-decision-tables}

\subsection{Evaluation of Scientific
Uncertainty}\label{evaluation-of-scientific-uncertainty}

\subsection{Regional management
considerations}\label{regional-management-considerations}

Widow Rockfish have shown latitudinal differences in life-history
parameters, which has led past assessment authors to pursue a two-area
model. Modelling a stock with two areas is difficult because it requires
many assumptions about recruitment distribution, movement, and
connectivity, while also splitting data into two areas that reduces
sample sizes when compared to a coastwide model. The upside is that it
can result in a better model that more accurately predicts regional
status. This assessment is a coastwide model because not enough is known
about the assumptions that would have to be made for a two-area model.

It is still important to consider regional differences when making
management decisions. Following recent cohorts through time with survey
data showed that older fish showed up in the north after younger fish
were observed in the south (Figure 2). This may indicate connectivity
between the north and the south and that this is truly one stock.
However, more investigation is needed.

Widow Rockfish are managed on a coastwide basis and observed more often
in the NWFSC WCGBT bottom trawl survey north of latitude 40° 10′ N.
Bottom trawl catches in California have historically been as large as in
Oregon and larger than in Washington, but recently catches in California
have been small. Rockfish Conservation Areas (RCAs) cover a significant
proportion of Widow Rockfish habitat, but a midwater trawl fishery is
beginning to re-develop that can fish in these areas. Future assessments
and management of Widow Rockfish may want to monitor where catches are
being taken to make sure that specific areas are not being
overexploited. In addition, research on the connectivity along the coast
as well as regional differences would help to inform the potential for
overfishing specific areas.

\subsection{Research and Data Needs}\label{research-and-data-needs-1}

\newpage{}

\subsection{Acknowledgements}\label{sec-acknowledgements}

\newpage{}

\subsection{References}\label{references}

\newpage{}

\subsection{Tables}\label{tables}

\newpage{}

\subsection{Figures}\label{figures}

\phantomsection\label{refs}
\begin{CSLReferences}{1}{0}
\bibitem[\citeproctext]{ref-bradburn_2003_2011}
Bradburn, M. J., A. A Keller, and B. H. Horness. 2011. {``The 2003 to
2008 {US} {West} {Coast} Bottom Trawl Surveys of Groundfish Resources
Off {Washington}, {Oregon}, and {California}: Estimates of Distribution,
Abundance, Length, and Age Composition.''} US Department of Commerce,
National Oceanic; Atmospheric Administration, National Marine Fisheries
Service.

\bibitem[\citeproctext]{ref-weinberg_2001_2002}
Weinberg, K. L., M. E. Wilkins, F. R. Shaw, and M. Zimmermann. 2002.
{``The 2001 {Pacific} {West} {Coast} Bottom Trawl Survey of Groundfish
Resources: Estimates of Distribution, Abundance and Length and Age
Composition.''} NOAA Technical Memorandum NMFS-AFSC-128. U.S. Department
of Commerce.

\end{CSLReferences}




\end{document}
